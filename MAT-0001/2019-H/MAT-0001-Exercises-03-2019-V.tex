\documentclass[a4paper,11pt]{article}

% Always keep your main language last
\usepackage[british,nynorsk,samin,norsk]{babel}

\usepackage{_sty/UiT}
\usepackage{_sty/IMS}
\usepackage{MAT-0001/MAT-0001}

%==============================================================================%
%                              COURSE INFORMATION                              %
%==============================================================================%


\UiTsetup{%
    Language = {norsk}, %auto = language defined by babel
    isExercise = {true}, % Is this an exercise? true / false
    ExerciseNumber = {3}, % Which exercise number is this?
    Solution = {true}, % Boolean (true/false) includes solutions
    SolutionShort = {auto}, % is used by \UiTcorrect and \UiTwrong
    CourseCode = {auto}, % auto = \courseCode (defined in courseCode.sty, e.g. MAT-1001.sty)
    CourseName = {auto}, % auto = \courseName (defined in courseCode.sty, e.g. MAT-1001.sty)
    Year = {2019}, % No auto setting for this, as it might change on recompilation
    Month = {9}, % No auto setting for this, as it might change on recompilation
    Day = {9},% No auto setting for this, as it might change on recompilation
    DurationDays = {4}, % Set the number of days the exam should last
}

%==============================================================================%
%                     OWN COMMANDS AND PACKAGES BELOW HERE                     %
%==============================================================================%

\begin{document}

%==============================================================================%
%                                 EXERCISE 03                                  %
%==============================================================================%

\frontpageUiT

\titlebox[norsk]{auto}{\Spraak}{auto}{\dateDuration}

%==============================================================================%
%                                SEKSJON 1.10                                  %
%==============================================================================%

\subsection*{Seksjon \textcolor{\UiTnumbercolor}{1}.\textcolor{\UiTnumbercolor}{3}}

%==============================================================================%
%                               PROBLEM 1.10.1                                 %
%==============================================================================%

\begin{problem}[1]
    Regn ut følgende summer
    \begin{subproblem}{3}
        \item $\displaystyle \sum_{n=0}^6 4n^2 \UiTcorrect*[=]{364}$
                \label{subproblem:MAT-0001-Problem-1.10.1.a}
        \item $\displaystyle \sum_{n=1}^{10} 1\UiTcorrect*[=]{10}$
                \label{subproblem:MAT-0001-Problem-1.10.1.b}
        \item $\displaystyle \sum_{j=0}^4 (5j^2 - 2)\UiTcorrect*[=]{140}$
                \label{subproblem:MAT-0001-Problem-1.10.1.c}
    \end{subproblem}
    \begin{subproblem}[4]{3}
        \item $\displaystyle \sum_{p=-1}^0 2p\UiTcorrect*[=]{-2}$
                \label{subproblem:MAT-0001-Problem-1.10.1.d}
        \item $\displaystyle \sum_{i=-100}^{100} i^3\UiTcorrect*[=]{0}$
        \label{subproblem:MAT-0001-Problem-1.10.1.e}
    \end{subproblem}
\end{problem}

\begin{solution}
    Her blir kanskje det enkleste å bruke formlene for summen av ulike rekker
    %
    \begin{equation*}
        \sum_{n=0}^N 1 = N+1, \qquad
        \sum_{n=0}^N n = \frac{N(N+1)}{2}, \qquad
        \sum_{n=0}^N n^2 = \frac{N(N+1)(2N+1)}{6}.
    \end{equation*}
    %
    \Opg{subproblem:MAT-0001-Problem-1.10.1.a} Ved å bruke formelene ovenfor får vi direkte
    %
    \begin{equation*}
           \sum_{n=0}^6 4n^2 
        = 4\sum_{n=0}^6 n^2 
        = 4 \cdot \frac{\textcolor{UiT-blue}{6}(6+1)(2\cdot 6+1)}{\textcolor{UiT-blue}{6}}
        = 4 (7) (13) = 52 \cdot 7 = 364.
    \end{equation*}
    %
    Siden vi har 52 uker i ett år, og syv dager i hver uke får vi totalt 364.
    \begin{flalign*}
     \Opg{subproblem:MAT-0001-Problem-1.10.1.b} &&
        \sum_{n=1}^{10} a_j 1 
        &= \underbrace{1 + 1 + \cdots + 1}_{\text{$10$ ganger}} = 10 && 
    \end{flalign*}
    \Opg{subproblem:MAT-0001-Problem-1.10.1.c} Vi har
    %
    \begin{align*}
             \sum_{j=0}^4 5j^2
        &= 5 \cdot \frac{(4) (4 + 1)(2\cdot 4 + 1)}{6}
         = 5 \cdot \frac{(2 \cdot 2)(5)(3 \cdot 3)}{3 \cdot 2}
         = 5 \cdot 2 \cdot 5 \cdot 3 
         = 150 \\
            \sum_{j=0}^{4} -2
        &= -2(4 + 1) 
         = -10 \\
        \sum_{j=0}^{4} 5j^2 - 2 &= 150 - 10 = 140,
    \end{align*}
    \begin{flalign*}
     \Opg{subproblem:MAT-0001-Problem-1.10.1.d} &&
        \sum_{p=-1}^{0} 2p 
        &= 2\cdot(-1) + 2\cdot(0) = -2. && 
    \end{flalign*}
    \Opg{subproblem:MAT-0001-Problem-1.10.1.e} Her har vi i utgangspunktet ingen formel, og
    det er heller ingen grunn til å bruke en. Dersom vi prøver med mindre verdier ser vi at
    %
    \begin{equation*}
        \sum_{i=-2}^{2} i^3 = (-2)^3 + (-1)^3 + (0)^3 + (1)^3 + (2)^3
        = -8 - 1 + 0 + 1 + 8 = 0.
    \end{equation*}
    %
    Hvor vi ser at alle de negative leddene kanselerer de positive. Det er rimelig å anta 
    at dette fortsetter for større verdier også. Mer formelt kan vi og føre det som følger
    %
    \begin{align*}
          \sum_{i=-N}^{N} i^3
        &= \sum_{i=-N}^{-1} i^3 + \sum_{i=0}^0 i^3 + \sum_{i=1}^{N} i^3 \\
        &= \sum_{-i=-N}^{1} (-i)^3 + 0 + \sum_{i=1}^{N} i^3 
         = -\sum_{i=1}^{N} i^3 + \sum_{i=1}^{N} i^3 
         = 0.
    \end{align*}
    %
\end{solution}

%==============================================================================%
%                               PROBLEM 1.10.2                                 %
%==============================================================================%

\begin{problem}
    \label{problem:MAT-0001-Problem-1.10.2}
    Skriv $\displaystyle \sum_{j=1}^5a_j 2^j$ uten bruk av summetegn. 
    $\UiTcorrect{2 a_1 + 4 a_2 + 8 a_3 + 16 a_4 + 32 a_5}$
\end{problem}

\begin{solution}
    \begin{flalign*}
     \Opg{problem:MAT-0001-Problem-1.10.2} &&
        \sum_{j=1}^5a_j 2^j 
        &= a_12^1 + a_22^2 + a_32^3 + a_42^4 + a_52^5 && \\ &&
        &= 2 a_1 + 4 a_2 + 8 a_3 + 16 a_4 + 32 a_5. &&
    \end{flalign*}
\end{solution}

%==============================================================================%
%                               PROBLEM 1.10.3                                 %
%==============================================================================%

\begin{problem}
    Skriv uttrykket
    %
    \begin{equation*}
        4^3 + 4^5 + 4^5 + 4^6 \UiTcorrect*[=]{\sum_{n=3}^6 4^n},
    \end{equation*}
    %
    ved bruk av summetegn.
\end{problem}

\newpageNotLF
\newpageLF

%==============================================================================%
%                                SEKSJON 1.X                                   %
%==============================================================================%

\subsection*{Blandede oppgavel til kapittel \textcolor{\UiTnumbercolor}{1}}

%==============================================================================%
%                               PROBLEM 1.X.4                                  %
%==============================================================================%

\begin{problem}[4]
    Forenke følgende uttrykk så mye du kan. Alle bokstaver står for reelle
    tall.
    %
    \begin{subproblem}{2}
        \item $\displaystyle \frac{x^3 + x^2 y}{x + x^2}$
        \label{subproblem:MAT-0001-Problem-1-x-4.a}
        \item $\displaystyle \frac{x^2 y^3 + y^4x}{xy^3 + y^4}$
        \label{subproblem:MAT-0001-Problem-1-x-4.b}
    \end{subproblem}
    \begin{subproblem}[3]{2}
        \item $\displaystyle \frac{1}{x - 1} - \frac{1}{x}$
        \label{subproblem:MAT-0001-Problem-1-x-4.c}
        \item $\displaystyle \frac{xy + 1 + xy^2}{x}$
        \label{subproblem:MAT-0001-Problem-1-x-4.d}
    \end{subproblem}
    \begin{subproblem}[5]{2}
        \item $\displaystyle \frac{yx - x}{y - 1}$
        \label{subproblem:MAT-0001-Problem-1-x-4.e}
        \item $\displaystyle \frac{rare}{greier}$
        \label{subproblem:MAT-0001-Problem-1-x-4.f}
    \end{subproblem}
    \begin{subproblem}[7]{2}
        \item $\displaystyle \biggr[3 - \frac{5x^2 tz + x^2 tz}{tx^2z}\biggl]^3$
        \label{subproblem:MAT-0001-Problem-1-x-4.g}
        \item $\displaystyle \frac{\text{Dette} - \text{er} - \text{egentlig}}{\text{bare} - \text{sludder}}$
        \label{subproblem:MAT-0001-Problem-1-x-4.h}
    \end{subproblem}
\end{problem}

\begin{solution}
    \begin{flalign*}
          \ref{subproblem:MAT-0001-Problem-1-x-4.a} &&
         & \frac{x^3 + x^2 y}{x + x^2}
         = \frac{x (x^2 + x y)}{x(1 + x)} 
         = \frac{x^2 + xy}{1 + x},& \\ 
           \ref{subproblem:MAT-0001-Problem-1-x-4.b} &&
         & \frac{x^2 y^3 + y^4x}{xy^3 + y^4}
         = \frac{x \cdot x y^3 + y \cdot xy^3}{x \cdot y^3 + y\cdot y^3}
         = \frac{xy^3(x + y)}{y^3(x + y)}
         = x, & \\
           \ref{subproblem:MAT-0001-Problem-1-x-4.c} &&
         & \frac{1}{x-1} - \frac{1}{x}
         = \frac{1}{x-1}\cdot\frac{x}{x} - \frac{1}{x}\frac{x-1}{x-1}
         = \frac{x - (x-1)}{x(x-1)}
         = \frac{1}{x(x-1)},& \\
           \ref{subproblem:MAT-0001-Problem-1-x-4.d} &&
         & \frac{1}{x-1} - \frac{1}{x}
         = \frac{1}{x-1}\cdot\frac{x}{x} - \frac{1}{x}\frac{x-1}{x-1}
         = \frac{x - (x-1)}{x(x-1)}
         = \frac{1}{x(x-1)},& \\
           \ref{subproblem:MAT-0001-Problem-1-x-4.e} &&
          & \frac{yx - x}{y - 1}
          = \frac{(y-1)x}{y-1}
          = \frac{y-1}{y-1}x
          = x, & \\
           \ref{subproblem:MAT-0001-Problem-1-x-4.f} &&
          & \frac{rare}{greier}
          = \frac{er^2\cdot a}{er^2\cdot egi}
          = \frac{er^2}{er^2} \cdot \frac{a}{egi}
          = \frac{a}{egi}. &
    \end{flalign*}
    \Opg{subproblem:MAT-0001-Problem-1-x-4.g} Alle leddene inneholder $x^2tz$ slik at
    \begin{equation*}
            \frac{5x^2 tz + x^2 tz}{tx^2z}
          = \frac{5x^2 tz}{x^2tz} + \frac{x^2 tz}{x^2tz}
          = 5 + 1 = 6,
    \end{equation*}
    som betyr at vår opprinnelige likning forenkles til 
    \begin{equation*}
        \biggr[3 - \frac{5x^2 tz + x^2 tz}{tx^2z}\biggl]^3
        = \biggr[3 - 6\biggl]^3
        = (-3)^3 = -27.
    \end{equation*}
    \Opg{subproblem:MAT-0001-Problem-1-x-4.g} ved å trekke ut de felles bokstavene fås
    %
    \begin{align*}
        \frac{\text{Dette} - \text{er} - \text{egentlig}}{\text{bare} - \text{sludder}}
        & = \frac{\text{D}\text{e}^2\text{t}^2 - \text{er} - \text{e}^2\text{g}^2\text{ilnt}}%
                 {\text{aber} - \text{d}^2\text{elsru}} \\
        & = \frac{\text{e}(\text{D}\text{e}\text{t}^2 - \text{r} - \text{e}\text{g}^2\text{ilnt})}%
                 {\text{e}(\text{abr} - \text{d}^2\text{lsru})} 
          = \frac{\text{D}\text{e}\text{t}^2 - \text{r} - \text{e}\text{g}^2\text{ilnt}}%
                 {\text{abr} - \text{d}^2\text{lsru}}.
    \end{align*}
\end{solution}

%==============================================================================%
%                               PROBLEM 1.X.5                                  %
%==============================================================================%

\begin{problem}
    Forenkle uttrykkene så mye du kan, og forklar underveis hvilke regneregler 
    du bruker. Alle bokstaver står for tall.
    \begin{subproblem}{2}
        \item $\displaystyle \frac{x + x^5}{x}$
          \label{subproblem:MAT-0001-Problem-1-x-5-a}
        \item $\displaystyle \frac{x^2 ya + (ax)^2 y}{ay}$
          \label{subproblem:MAT-0001-Problem-1-x-5-b}
    \end{subproblem}
    \begin{subproblem}[3]{2}
        \item $\displaystyle x^2 + x^3 + 3x + 2x$
          \label{subproblem:MAT-0001-Problem-1-x-5-c}
        \item $\displaystyle (5a + 2)(a - 4) - 5a^2$
          \label{subproblem:MAT-0001-Problem-1-x-5-d}
    \end{subproblem}
    \begin{subproblem}[5]{2}
        \item $\displaystyle (2x^2 + 8) \cdot \frac{x+3}{2}$
          \label{subproblem:MAT-0001-Problem-1-x-5-e}
        \item $\displaystyle \frac{x}{(x^2 - 2x)(x + 1)}$
          \label{subproblem:MAT-0001-Problem-1-x-5-f}
    \end{subproblem}
\end{problem}

\begin{solution}
    \Opg{subproblem:MAT-0001-Problem-1-x-5-a} siden alle leddene i teller og nevner
    inneholder $x$ kan vi forkorte
    %
    \begin{equation}
        \frac{x + x^5}{x} = \frac{x(1+x^4)}{x} = 1 + x^4.
    \end{equation}
    %
    \Opg{subproblem:MAT-0001-Problem-1-x-5-b} Alle leddene inneholder $ay$
    %
    \begin{align*}
           \frac{x^2 ya + (ax)^2 y}{ay}
        &= \frac{ay(x^2 + ax^2)}{ay}
         = (1 + a)x^2, \ \text{eller} \\
           \frac{x^2 ya + (ax)^2 y}{ay}
        &= \frac{x^2 ya}{ay} + \frac{x^2 ay}{ay}
         = (1 + a)x^2.
    \end{align*}
    \Opg{subproblem:MAT-0001-Problem-1-x-5-c} Ikke veldig mye som kan gjøres her
    %
    \begin{equation*}
        x^2 + x^3 + 3x + 2x = x^3 + x^2 + 5x = x(x^2 + x + 4).
    \end{equation*}
    \Opg{subproblem:MAT-0001-Problem-1-x-5-d} Ved å gange ut fås
    %
    \begin{equation*}
        (5a + 2)(a - 4) - 5a^2
        = (5a^2 + 2a - 20a - 8) - 5a^2 
        = -18a - 8.
    \end{equation*}    
    \Opg{subproblem:MAT-0001-Problem-1-x-5-e} Merk at $2$ deler både $8$ og $2x^2$
    \begin{equation*}
          (2x^2 + 8) \cdot \frac{x+3}{2}
        = 2(x^2 + 4) \cdot \frac{x+3}{2}
        =  (x^2 + 4)(x + 3).
    \end{equation*}
    \Opg{subproblem:MAT-0001-Problem-1-x-5-f}
    \begin{equation*}
          \frac{x}{(x^2 - 2x)(x+1)}
        = \frac{x}{(x\cdot x - 2\cdot x)(x+1)}
        = \frac{x}{x(x - 2)(x+1)}
        = \frac{1}{(x-2)(x+1)}.
    \end{equation*}
\end{solution}

\newpageLF

%==============================================================================%
%                               PROBLEM 1.X.7                                  %
%==============================================================================%

\begin{problem}[7]
    Finn eventuelle løsninger til følgende likninger
    %
    \begin{subproblem}{2}
        \item $2x^2 + 3x - 1 = 0$
        \label{subproblem:MAT-0001-Exercise-03-Problem-1-x-7-a}
        \item $16x^2 - 16x + 5 = 0$
        \label{subproblem:MAT-0001-Exercise-03-Problem-1-x-7-b}
    \end{subproblem}
    \begin{subproblem}[3]{2}
        \item $N^2 + 3N - 8 = 0$
        \label{subproblem:MAT-0001-Exercise-03-Problem-1-x-7-c}
        \item $\lambda^2 - 2\lambda + 6 = 0$
        \label{subproblem:MAT-0001-Exercise-03-Problem-1-x-7-d}
    \end{subproblem}
\end{problem}

\begin{solution}
    \Opg{subproblem:MAT-0001-Exercise-03-Problem-1-x-7-a}
    Her finnes det dessverre ingen smarte triks: 
    %
    \begin{equation*}
        2x^2 + 3x - 1 = 0 \ \Leftrightarrow \
        x = \frac{-3 \pm \sqrt{3^2 - 4\cdot 2 \cdot (-1)}}{2\cdot 2}
          = - \frac{3}{4} \pm \frac{\sqrt{17}}{4}.
    \end{equation*}
    \Opg{subproblem:MAT-0001-Exercise-03-Problem-1-x-7-b}
    Skal vi være litt for smart for vårt eget beste legg merke til at 
    $16x^2 - 16x + 5 = 16x(x-1) + 5$. Den minste verdien som $x(x-1)$ kan oppnå
    er $-1/4$ når $x=1/2$. Dette betyr at vi har
    %
    \begin{equation*}
        16x(x-1) + 5 \geq 16 \cdot \bigl( -\frac{1}{4}\bigr) + 5 = 1 > 0,
    \end{equation*}
    %
    slik at polynomet ikke har noen nullpunkter. Hvorfor oppnår $x(x-1)$ sitt 
    maksimum for $x=1/2$? Man kan bruke mekaniske derivasjon, ellers litt mer 
    intuitivt anta vi har ett rektangel med sider $x$ og $x-1$. Dette rektanglet 
    har størst areal når det er et kvadrat, som vil si at vi ønsker at $x$ og 
    $x-1$ skal være like lange modulo fortegn. Ellers kan vi og fullføre 
    kvadratet og se at $x(x-1) = (x-1/2)^2 - 1/4$. For en siste mulighet kan 
    vi og drøfte diskriminanten til det opprinnelige polynomet
    %
    \begin{equation}
        \Delta = b^2 - 4ac = 16^2 - 4 \cdot 16 \cdot 5 = 16 (16 - 20) < 0.
    \end{equation}
    \medskip
    \Opg{subproblem:MAT-0001-Exercise-03-Problem-1-x-7-c} Her ønsker vi på undersøke om det eksisterer
    to heltall slik at $a+b=3$ og $ab = 8$. Alle tallpar der produktet gir $8$ er $(1,8)$, $(2,4)$, $(-1,-8)$ og $(-2,-4)$. 
    Ingen av disse gir $3$ når vi legger dem sammen, slik at løsningen ikke har noen heltallsrøtter. 
    Vi må derfor ty til siste mulighet, andregradsformelen
    %
    \begin{equation*}
        N^2 + 3N - 8 = 0
        \ \Leftrightarrow \ 
        N = \frac{-3 \pm \sqrt{3^2 - 4\cdot 1 \cdot (-8)}}{2\cdot 1}
          = -\frac{3}{2} \pm \frac{\sqrt{41}}{2}.
    \end{equation*}
    \medskip
    \Opg{subproblem:MAT-0001-Exercise-03-Problem-1-x-7-d} Slik som før
    så er den minste verdien som $\lambda(\lambda - 2)$ kan få er $-1$ når $\lambda = 1$. 
    Fordi da er $\lambda$ og $\lambda - 2$ like lange i absoluttverdi, og danner da ett kvadrat. Dermed så er 
    %
    \begin{equation*}
        \lambda^2 - 2\lambda + 6 
        = \lambda(\lambda - 2) + 6
        \geq -1 + 6 \geq 5 > 0,
    \end{equation*}
    %*
    og polynomet har ingen reell løsning. Alternativt via andregradsformelen
    %
    \begin{equation*}
        \lambda^2 - 2\lambda + 6 
        \ \Leftrightarrow \ 
        \lambda 
        = \frac{2 \pm \sqrt{(-2)^2 - 4(1)(6)}}{2}
        = \frac{2 \pm \sqrt{-2^2\cdot 5}}{2} 
        = 1 \pm i\sqrt{5}.
    \end{equation*}
\end{solution}



%==============================================================================%
%                               PROBLEM 1.X.8                                  %
%==============================================================================%

\begin{problem}
    For hvilke verdier av $k$ har
    %
    \begin{equation*}
        \lambda^2 - k \lambda + 9 = 0,
    \end{equation*}
    %
    to forskjellige reelle røtter. 
\end{problem}

\begin{solution}
Kanskje den enkleste måten er at polynomet har to forskjellige løsninger
når diskriminanten $\Delta =  b^2 - 4ac$ er strengt positiv. Innsetning gir
%
\begin{equation*}
    \Delta > 0 
    \, \Rightarrow \, 
    b^2 - 4ac  > 0 
    \, \Rightarrow \, 
    (-k)^2 - 4\cdot 1 \cdot 9> 0 
    \, \Rightarrow \, 
    k^2 > 6^2.
\end{equation*}
%
Hvor vi da ser at vi trenger at $\abs{k} > 6$, som er det samme som at vi 
enten må ha $k < -6$ eller $k > 6$. Men hvorfor fungerer dette? La oss sjekke 
med andregradsformelen
%
\begin{equation}
    \lambda = \frac{-k \pm \sqrt{(-k)^2 - 4\cdot 1 \cdot 9}}{2}
     = \frac{-k \pm \sqrt{\Delta}}{2},
\end{equation}
%
hvor vi gjenkjente det under rottegnet som diskriminanten. Det er nå klart at 
vi har ingen løsning når $\Delta < 0$, én løsning når $\Delta = 0$ og 
\emph{to} løsninger når $\Delta > 0$.
\end{solution}

%==============================================================================%
%                               PROBLEM 1.X.9                                  %
%==============================================================================%

\begin{problem} 
    La $L \subseteq \R^2$ være linjen $y = 1 - (1/2)x$.
    \begin{subproblem}
        \label{subproblem:MAT-0001-Exercise-03-Problem-1-x-9-a}
        Finn likningen for linjen som er \emph{parallell} med $L$ og som går 
        gjennom punktet $(1, 3)$.
    \end{subproblem}
    \begin{subproblem}
        \label{subproblem:MAT-0001-Exercise-03-Problem-1-x-9-b}
        Finn likningen for linjen som står \emph{vinkelrett} på $L$ og som går 
        gjennom punktet $(1, 3)$.
    \end{subproblem}
\end{problem}

\begin{solution}
    \Opg{subproblem:MAT-0001-Exercise-03-Problem-1-x-9-a} At 
    vår linje skal være \emph{parallell} med $y = 1 - (1/2)x$ betyr at 
    de må ha samme \emph{stigningstall}. Hva er stignignstallet til den opprinelige
    linja? Jo, $-1/2$. Dermed må vår linje være på formen
    %
    \begin{equation*}
        y = b - \frac{1}{2}x.
    \end{equation*}
    %
    For å bestemme konstanten $b$ bruker vi at linja passerer igjennom $(1, 3)$
    %
    \begin{equation*}
        3 = b - \frac{1}{2} \cdot 1 
        \ \Rightarrow \ 
        b = 3 + \frac{1}{2}
        \ \Rightarrow \ 
        b = \frac{7}{2}.
    \end{equation*}
    %
    Formelen for en linje som går igjennom $(1, 3)$ \emph{og} er parallell med 
    $y = 1 - (1/2)x$ er altså $y = (7/2) - (1/2)x = (7 - x)/2$.
    \medskip
\end{solution}

\begin{solution}
    \Opg{subproblem:MAT-0001-Exercise-03-Problem-1-x-9-b} At 
    vår linje skal stå \emph{vinkelrett}\footnote{Fjonge matematikere kaller gjerne dette for \emph{ortogonalitet}.} 
    på $y = 1 - (1/2)x$ betyr at \emph{produktet} av stigningstallene skal bli $-1$. 
    Så dersom linjen vår er på formen $y = ax$ så trenger vi at 
    %
    \begin{equation*}
        -\frac{1}{2}a = - 
        \ \Rightarrow \ 
        a = 2.
    \end{equation*}
    %
    For å bestemme konstanten $b$ bruker vi at linja passerer igjennom $(1, 3)$
    %
    \begin{equation*}
        3 = b + 2 \cdot 1 
        \ \Rightarrow \ 
        b = 3 - 2
        \ \Rightarrow \ 
        b = 1
    \end{equation*}
    %
    Formelen for en linje som går igjennom $(1, 3)$ \emph{og} står ortogonalt på 
    $y = 1 - (1/2)x$ er altså $y = 5 - 2x$. Her har vi tegnet linjene i \cref{fig:MAT-0001-Problem-1-x-9}
    for å forsikre oss om at vi har regnet riktig.
\end{solution}

\hvisLF{%
  \begin{figure}[htbp!]
    \centering
    \begin{tikzpicture}
      \begin{axis}[
        UiTplotstyle,
        ytick={-5,-3,...,5},
        xtick={-5,-3,...,5},
        % x tick label style={below right},
        % y tick label style={above left,},
        ymin=-2.75,
        ymax=4.5,
        xmin=-4.5,
        xmax=4.5,
        domain=-10:10,
        legend pos = south west,
        legend cell align={left},
        ]
        \addplot+[domain=-5:5, mark=none, ultra thick, black] {1-0.5*x};
        \addplot+[domain=-5:5, mark=none, ultra thick, UiT-blue] {3.5 - 0.5*x};
        \addplot+[domain=-5:5, ultra thick,mark=none, UiT-orange] {1 + 2*x};
        \legend{$y = 1 - (1/2)x$, $y = (7-x)/2$, $y = 1 + 2x$}
      \end{axis}
    \end{tikzpicture}
    \caption{}
    \label{fig:MAT-0001-Problem-1-x-9}. 
  \end{figure}
}

\newpageNotLF

%==============================================================================%
%                                NOTAT 2.6.6                                   %
%==============================================================================%

\subsection*{Forelesningsnotater: seksjon \textcolor{\UiTnumbercolor}{2}.\textcolor{\UiTnumbercolor}{6}.\textcolor{\UiTnumbercolor}{6}}

\begin{align}
  \langle&a\rangle
  = \brak{a_1}^{p_1}\brak{a_2}^{p_2}\dots\brak{a_n}^{p_n},
  \label{Mregel}\\
  [&a]
  = \,\bpar{a_1}^{p_1}\,\bpar{a_2}^{p_2}\,\dots\,\bpar{a_n}^{p_n}
  \label{Bregel}.
\end{align}

Vi har altså at for enhver benevnt avledet
størrelse $a$, så finnes det potenser $q_1,\dots,q_k$ slik at
%
\begin{equation}
  \bpar{a} = d_1^{q_1}\cdots d_k^{q_k},\label{BasicUnitExpansion}
\end{equation}
%
hvor $d_1,\cdots,d_k$ er kortnavnene for enhetene assosierte med de benevnte
grunnstørrelsene.

%==============================================================================%
%                           NOTAT 2.6.6 | PROBLEM 1                            %
%==============================================================================%

\begin{problem}[1]
    La følgene benevnte størrelser være gitt
    \begin{align*}
      a_1 &= \SI{3.6}{\m\squared\s\tothe{\frac{1}{2}}},\\
      a_2 &= \SI{5.1}{\m\tothe{\frac{1}{3}}\s\tothe{-4}},
    \end{align*}
    og la $a$ være den benevnte avledede størrelsen
    \begin{equation*}
      a = a_1^{\frac{1}{3}}a_2^2.
    \end{equation*}
\end{problem}
\begin{subproblem}
    Finn måltallet til $a$ ved å benytte
    regelen \eqref{Mregel} for å finne måltallet til en benevnt avledet
    størrelse.
\end{subproblem}
\begin{solution}
    Ved å bruke \cref{Mregel} 
    %
    \begin{equation}
          \brak{a} 
        = \brak{a_1}^{\frac{1}{3}}\brak{a_2}^2
        = \num{3.6}^{\frac{1}{3}}\num{5.1}^2
        \approx \num{39.8634},
    \end{equation}
    %
    så er måltallet vårt altså $\brak{a}=\num{39.8634}$.
\end{solution}
\begin{subproblem}
    Vis, ved å benytte regelen \eqref{Bregel} for å finne
    benevningen til en benevnt avledet størrelse, at benevningen til
    $a$ kan skrives på formen \eqref{BasicUnitExpansion}. Angi hva
    $k$, $d_1,\dots,d_k$ og $p_1,\dots, p_k$ er i dette tilfellet.
\end{subproblem}
\begin{solution}
    Ved å bruke \cref{Bregel}
    %
    \begin{align*}
            \bpar{a} 
          = \bpar{a_1}^{\frac{1}{3}}\bpar{a_2}^2
        & = (\si{\m\squared\s\tothe{\frac{1}{2}}})^{\frac{1}{3}}
            (\si{\m\tothe{\frac{1}{3}}\s\tothe{-4}})^2 \\
        & = \si{\m\tothe{2\cdot\frac{1}{3}+\frac{1}{3}\cdot 2}%
                \s\tothe{\frac{1}{2}\cdot\frac{1}{3}-4\cdot 2}}
          = \si{\m\tothe{\frac{4}{3}}\s\tothe{-\frac{47}{6}}}
    \end{align*}
    %
    Her er $d_1 = \si{m}$, $p_1 = \frac{3}{4}$ og $d_2 = \si{\s}$, $p_2 = -\frac{47}{6}$, $k=2$.
    Ved å kombinere får vi altså at $a = \SI{39.8634}{\m\tothe{\frac{4}{3}}\s\tothe{-\frac{47}{6}}}$. 
\end{solution}
\begin{subproblem}
    Skift fra gamle enheter til nye enheter ved at
    \begin{align*}
      \si{\s} &\rightarrow \si{\hour},\\
      \si{\m} &\rightarrow \si{\km}.
    \end{align*}
    Beregn måltall og benevning til $a$ med hensyn på de nye
    enhetene på to måter
    \begin{subsubproblem}
        \label{subsubproblem:MAT-0001-Exercise-03-Problem-N-2.6.6.1-c-i}
        Skift først enheter i $a_1$ og $a_2$ og finn de nye måltallene
        og benevningene tilhørende disse to benevnte størrelsene. Finn så
        nytt måltall og benevning til $a$ ved å bruke reglene fra \cref{Mregel,Bregel}.
    \end{subsubproblem}
    \begin{subsubproblem}
        \label{subsubproblem:MAT-0001-Exercise-03-Problem-N-2.6.6.1-c-ii}
        Bruk måltallet og benevningen til $a$ med hensyn på gamle
        enheter, funnet i oppgavene a og b, til å direkte beregne måltallet
        og benevningen til $a$ med hensyn på de nye enhetene.
    \end{subsubproblem}
\end{subproblem}

\begin{solution}
    \Opg{subsubproblem:MAT-0001-Exercise-03-Problem-N-2.6.6.1-c-i}
    Ved å skifte enheter har vi
    \begin{alignat*}{2}
      a_1 &= \num{3.6} \cdot (\SI{e-3}{\km})^{2}
                             (\SI[quotient-mode=fraction]{1/3600}{\hour})^{\frac{1}{2}}
         &&= \SI{6e-8}{\km\squared\hour\tothe{\frac{1}{2}}}, \\
      a_2 &= \num{5.1} \cdot (\SI{e-3}{\km})^{\frac{1}{3}}
                             (\SI[quotient-mode=fraction]{1/3600}{\hour})^{-4}
          &&= \SI{8.5660416e13}{\km\tothe{\frac{1}{3}}\hour\tothe{-4}}
    \end{alignat*}
    % 
    Ved nå og bruke reglene fra \cref{Mregel,Bregel} så blir henholdsvis måltallet og benevningen til $a$
    %
    \begin{align*}
        \brak{a} &= \brak{a_1}^{\frac{1}{3}}\brak{a_2}^2 \\
                 &= (\num{6e-8})^{\frac{1}{3}}(\num{8.5660416e13})^2 
                  \approx \num{2.872615e25} \\
        \bpar{a} &= \bpar{a_1}^{\frac{1}{3}}\bpar{a_2}^2 \\
                 &= (\si{\km\squared\hour\tothe{\frac{1}{2}}})^{\frac{1}{3}}
                    (\si{\km\tothe{\frac{1}{3}}\hour\tothe{-4}})^2 
                  = \si{\km\tothe{2\cdot\frac{1}{3}+\frac{1}{3}\cdot 2}%
                        \hour\tothe{\frac{1}{2}\cdot\frac{1}{3}-4\cdot 2}} 
                  = \si{\km\tothe{\frac{4}{3}}\hour\tothe{-\frac{47}{6}}}.
    \end{align*}
    %
    Ved å kombinere likningene ovenfor får vi
    %
    \begin{equation*}
        a \approx \SI{2.872615e25}{\km\tothe{\frac{4}{3}}\hour\tothe{-\frac{47}{6}}}
    \end{equation*}
    %
    \Opg{subsubproblem:MAT-0001-Exercise-03-Problem-N-2.6.6.1-c-ii}
    Ved å direkte endre på $a$ får vi
    %
    \begin{align*}
        a &= \SI{39.8634}{\m\tothe{\frac{4}{3}}\s\tothe{-\frac{47}{6}}}\\
          &= 39.8634 \cdot (\SI{e-3}{\km})^{\frac{4}{3}} 
                           (\SI[quotient-mode=fraction]{1/3600}{\hour})^{-\frac{47}{6}} 
           \approx \SI{2.872615e25}{\km\tothe{\frac{4}{3}}\hour\tothe{-\frac{47}{6}}}
    \end{align*}
    %
    Hvor vi ser at vi fikk det samme svaret som før. 
\end{solution}

%==============================================================================%
%                           NOTAT 2.6.6 | PROBLEM 2                            %
%==============================================================================%

\begin{problem}
    Gjenta oppgave 1 med følgende benevnte størrelser
    \begin{align*}
      a_1 &= \SI{6.3}{\m\tothe{\frac{1}{4}}\kg\cubed},\\
      a_2 &= \SI{31.1}{\m\tothe{-\frac{1}{3}}\kg\tothe{-\frac{1}{2}}},
    \end{align*}
    hvor $a$ nå er
    \begin{equation*}
      a = a_1^3 a_2^{\frac{3}{2}}
    \end{equation*}
    og hvor vi har følgende skifte fra gamle til nye enheter
    \begin{align*}
      \si{\kg} &\rightarrow \si{\g},\\
      \si{\m} &\rightarrow \si{\milli\m}.
    \end{align*}
\end{problem}
\begin{solution}
    Ved å bruke \cref{Mregel} får vi med en gang at 
    %
    \begin{equation*}
          \brak{a} 
        = \brak{a_1}^{3}\brak{a_2}^{\frac{3}{2}}
        = \num{6.3}^{3}\num{31.1}^{\frac{3}{2}}
        \approx \num{4.37863e4}
    \end{equation*}
    Ved å bruke \cref{Bregel} får vi med en gang at 
    %
    \begin{align*}
            \bpar{a} 
          = \bpar{a_1}^{3}\bpar{a_2}^{\frac{3}{2}}
        & = (\si{\m\tothe{\frac{1}{4}}\kg\cubed})^{3}
            (\si{\m\tothe{-\frac{1}{3}}\kg\tothe{-\frac{1}{2}}})^{\frac{3}{2}} \\
        & = \si{\m\tothe{\frac{1}{4}\cdot 3 - \frac{1}{3}\cdot \frac{3}{2}}%
                \kg\tothe{3\cdot 3 - \frac{1}{2} \cdot \frac{3}{2}}}
          = \si{\m\tothe{\frac{1}{4}}\kg\tothe{\frac{33}{4}}}
    \end{align*}
    Slik at vi ser at vi har
    \begin{equation*}
        a = a_1^3 a_2^{\frac{3}{2}} 
          \approx \SI{4.37863e4}{\m\tothe{\frac{1}{4}}\kg\tothe{\frac{33}{4}}}
    \end{equation*}
\end{solution}
\begin{solution}
    \Opg{subsubproblem:MAT-0001-Exercise-03-Problem-N-2.6.6.1-c-i}
    Ved å skifte enheter har vi
    \begin{alignat*}{2}
      a_1 &= \phantom{3}\num{6.3} \cdot (\SI{e3}{\milli\m})^{\frac{1}{4}}
                             (\SI{e3}{\g})^{3}
          &&= \SI{3.54275e10}{\milli\m\tothe{\frac{1}{4}}\g\cubed}, \\
      a_2 &= \num{31.1} \cdot (\SI{e3}{\milli\meter})^{-\frac{1}{3}}
                             (\SI{e3}{\g})^{-\frac{1}{2}}
          &&= \SI{9.898e-2}{\milli\meter\tothe{-\frac{1}{3}}\g\tothe{-\frac{1}{2}}}.
    \end{alignat*}
    % 
    Ved nå og bruke reglene fra \cref{Mregel,Bregel} så blir henholdsvis måltallet og benevningen til $a$
    %
    \begin{align*}
        \brak{a} &= \brak{a_1}^{3}\brak{a_2}^{\frac{3}{2}} \\
                 &= (\num{3.54275e10})^{3}(\num{9.898e-2})^{\frac{3}{2}} 
                  \approx \num{1.38464e30}, \\
        \bpar{a} &= \bpar{a_1}^{3}\bpar{a_2}^{\frac{3}{2}} \\
                 &= (\si{\milli\m\tothe{\frac{1}{4}}\g\cubed})^{3}
                    (\si{\milli\m\tothe{-\frac{1}{3}}\g\tothe{-\frac{1}{2}}})^{\frac{3}{2}} \\
                & = \si{\milli\m\tothe{\frac{1}{4}\cdot 3 - \frac{1}{3}\cdot \frac{3}{2}}%
                        \g\tothe{3\cdot 3 - \frac{1}{2} \cdot \frac{3}{2}}}
                  = \si{\milli\m\tothe{\frac{1}{4}}\g\tothe{\frac{33}{4}}}.
    \end{align*}
    %
    Da er det klart at
    %
    \begin{equation}
        a \approx \SI{1.38464e30}{\milli\m\tothe{-\frac{1}{4}}\g\tothe{\frac{33}{4}}}
    \end{equation}
    %
    \Opg{subsubproblem:MAT-0001-Exercise-03-Problem-N-2.6.6.1-c-ii}
    Derimot om vi endrer direkte på $a$
    %
    \begin{align*}
        a &= \SI{4.37863e4}{\m\tothe{\frac{1}{4}}\kg\tothe{\frac{33}{4}}}\\
          &= \num{4.37863e4} \cdot (\SI{e3}{\milli\m})^{\frac{1}{4}} 
                           (\SI{e3}{\g})^{\frac{33}{4}} \\
          &\approx
          \SI{1.38464e30}{\milli\m\tothe{\frac{1}{4}}\g\tothe{\frac{33}{4}}}
    \end{align*}
    %
    ser at vi får det samme svaret som før. 
\end{solution}


%==============================================================================%
%                           NOTAT 2.6.6 | PROBLEM 5                            %
%==============================================================================%

\begin{problem}[5]
    La følgende benevnte størrelser være gitt
    \begin{align*}
      a_1 &= \SI{1.2}{\kg\tothe{\frac{1}{3}}%
                                \m\tothe{4}%
                                \s\tothe{-\frac{5}{6}}},\\
      a_2 &= \SI{4.3}{\per\kg\cubed%
                                \m\tothe{\frac{4}{5}}%
                                \s\tothe{-\frac{1}{2}}},\\
      a_3 &= \SI{5.2}{\kg\tothe{\frac{1}{2}}%
                                \m\squared%
                                \s\tothe{\frac{3}{4}}},\\
      a_4 &= \SI{2.4}{\kg\squared%
                                \m\cubed%
                                \s\tothe{\frac{1}{2}}},
    \end{align*}
    og la $a$ være størrelsen
    \begin{equation*}
      a = a_1%
          a_2^{-\frac{10}{9}}%
          a_3^{\frac{86}{45}}%
          a_4^{-\frac{104}{45}}.
    \end{equation*}
    Vis at dersom vi endrer enheter ved at
    \begin{align*}
      \si{\m} &\rightarrow \si{\km},\\
      \si{\s} &\rightarrow \si{\hour},\\
      \si{\kg} &\rightarrow \si{\ct},
    \end{align*}
    så endres ikke måltallet for størrelsen. Vis så, ved å finne benevningen til
    $a$, at den er en ubenevnt avledet størrelse.
\end{problem}

\begin{solution}
    Vi begynner med å regne ut måltallet til $\alpha$ først
    %
    \begin{align*}
         \brak{a} 
      &= \brak{a_1}%
         \brak{a_2}^{-\frac{10}{9}}%
         \brak{a_3}^{\frac{86}{45}}%
         \brak{a_4}^{-\frac{104}{45}} \\ 
      &= (\num{1.2})%
         (\num{4.3})^{-\frac{10}{9}}%
         (\num{5.2})^{\frac{86}{45}}%
         (\num{2.4})^{-\frac{104}{45}} \\
      &= \num{7.32789e-1}.
    \end{align*}
    La oss først regne ut det nye måltallet
    \begin{alignat*}{2}
      \brak{a_1} &= (\num{1.2})(\num{0.5e4})^{\frac{1}{3}}
                    (\num{e-3})^4\biggl(\frac{1}{3600}\biggr)^{-\frac{5}{6}} 
        &&\approx \num{1.88693e-8},\\
      \brak{a_2} &= (\num{4.3})(\num{0.5e4})^{-3}
                    (\num{e-3})^{\frac{4}{5}}\biggl(\frac{1}{3600}\biggr)^{-\frac{1}{2}} 
      &&\approx \num{8.21693e-12},\\
      \brak{a_3} &= (\num{5.2})(\num{0.5e4})^{\frac{1}{2}}
                    (\num{e-3})^2\biggl(\frac{1}{3600}\biggr)^{\frac{3}{4}} 
      &&\approx \num{7.91155e-7},\\
      \brak{a_4} &= (\num{2.4})(\num{0.5e4})^{2}
                    (\num{e-3})^3\biggl(\frac{1}{3600}\biggr)^{\frac{1}{2}} 
        &&\approx \num{1.00000e-3}.
    \end{alignat*}
    Vi kan nå regne ut måltallet til $a$ med nye enheter
    %
    \begin{align*}
         \brak{a} 
      &= \brak{a_1}%
         \brak{a_2}^{-\frac{10}{9}}%
         \brak{a_3}^{\frac{86}{45}}%
         \brak{a_4}^{-\frac{104}{45}} \\ 
      &= (\num{1.88693e-8})%
         (\num{8.21693e-12})^{-\frac{10}{9}}%
         (\num{7.91155e-7})^{\frac{86}{45}}%
         (\num{e-3})^{-\frac{104}{45}} \\
      &= \num{7.32789e-1}.
    \end{align*}
    Slik at vi ser at måltallet $a$ ikke har endret seg. La oss nå bestemme
    benevningen til $a$ direkte
    %
    \begin{align*}
         \bpar{a}
      &= \bpar{a_1}%
         \bpar{a_2}^{-\frac{10}{9}}%
         \bpar{a_3}^{\frac{86}{45}}%
         \bpar{a_4}^{-\frac{104}{45}} \\
      &= (\si{\kg\tothe{\frac{1}{3}}%
              \m\tothe{4}\s\tothe{-\frac{5}{6}}})%
         (\si{\per\kg\cubed%
              \m\tothe{\frac{4}{5}}
              \s\tothe{-\frac{1}{2}}})^{-\frac{10}{9}}%
         (\si{\kg\tothe{\frac{1}{2}}%
              \m\squared%
              \s\tothe{\frac{3}{4}}})^{\frac{86}{45}}%
         (\si{\kg\squared%
              \m\cubed%
              \s\tothe{\frac{1}{2}}})^{-\frac{104}{45}}\\
      &= \si{\kg\tothe{(\frac{1}{3} + 3 \cdot \frac{10}{9} + \frac{1}{2}\frac{86}{45} - 2 \cdot \frac{104}{45})}%
             \m\tothe{(4-\frac{4}{5}\frac{10}{9}+2\frac{86}{45}-3\cdot \frac{104}{45})}%
             \s\tothe{(-\frac{5}{6}+\frac{1}{2}\frac{10}{9} + \frac{3}{4} \cdot \frac{86}{45} - \frac{1}{2}\frac{104}{45})}} \\
      &= \si{\kg\tothe{0}\m\tothe{0}\s\tothe{0}} = 1.
    \end{align*}
    %
    Altså er $a$ en ubenevnt størrelse som var det som skulle vises.
\end{solution}

\newpageNotLF

%==============================================================================%
%                           NOTAT 2.6.6 | PROBLEM 6                            %
%==============================================================================%

\begin{problem}[6]
    La følgende benevnte størrelser være gitt
    \begin{align*}
      a_1 &= \SI{6.5}{\m\squared\s\tothe{-\frac{5}{7}}},\\
      a_2 &= \SI{3.0}{\m\tothe{\frac{3}{5}}%
                                \s\tothe{-\frac{1}{3}}},\\
      a_3 &= \SI{7.3}{\m\tothe{\frac{2}{3}}%
                                \s\tothe{\frac{1}{4}}},\\
      a_4 &= \SI{8.0}{\m\cubed\s\tothe{-\frac{1}{2}}},
    \end{align*}
    og la $a$ og $b$ være følgende størrelser
    \begin{align*}
      a &=
      a_1^{91}%
      a_3^{96}%
      a_4^{-82},\\
      b &=
      a_2^{5}%
      a_3^{\frac{42}{13}}%
      a_4^{-\frac{67}{39}}.
    \end{align*}
    Vis at dersom vi endrer enheter ved at
    \begin{align*}
      \si{\m} &\rightarrow \si{\km},\\
      \si{\s} &\rightarrow \si{\hour},
    \end{align*}
    så endres ikke måltallet for størrelsene $a$ og $b$. Vis så ved å finne
    benevningen til $a$ og $b$ at de er ubenevnte avledede størrelser.
\end{problem}

\begin{solution}
    La oss først regne ut måltallet for henholdsvis $a$ og $b$
    %
    \begin{alignat*}{3}
        \bpar{a} 
         &= \bpar{a_1}^{91}\bpar{a_3}^{96} \bpar{a_4}^{-82} 
        &&= (\num{6.5})^{91} (\num{7.3})^{96} (\num{8.0})^{-82} 
        &&= \num{6.32020e82}, \\
        \bpar{b} 
         &= \bpar{a_2}^{5}\bpar{a_3}^{\frac{42}{13}} \bpar{a_4}^{-\frac{67}{39}} 
        &&= (\num{6.5})^{5}(\num{7.3})^{\frac{42}{13}} (\num{8.0})^{-\frac{67}{39}} 
        &&= \num{4.20089e3}.
    \end{alignat*}
    %
    Vi regner så om de benevnte størrelsene
    %
    \begin{alignat*}{3}
      a_1 &= \SI{6.5}{\m\squared\s\tothe{-\frac{5}{7}}}
          &&= (\num{6.5})(\SI{e-3}{\km})^2
                        (\SI[quotient-mode=fraction]{1/3600}{\hour})^{-\frac{5}{7}}
          &&\approx \SI{2.25491e-3}{\km\squared\hour\tothe{-\frac{5}{7}}},\\
      a_2 &= \SI{3.0}{\m\tothe{\frac{3}{5}}%
                      \s\tothe{-\frac{1}{3}}}
          &&= (\num{3.0})(\SI{e-3}{\km})^{\frac{3}{5}}
                        (\SI[quotient-mode=fraction]{1/3600}{\hour})^{-1/3}
          &&\approx \SI{7.28711e-1}{\km\tothe{\frac{3}{5}}\hour\tothe{-\frac{1}{3}}},\\
      a_3 &= \SI{7.3}{\m\tothe{\frac{2}{3}}%
                      \s\tothe{\frac{1}{4}}}
          &&= (\num{7.3})(\SI{e-3}{\km})^{\frac{2}{3}}
                        (\SI[quotient-mode=fraction]{1/3600}{\hour})^{\frac{1}{4}}
          &&\approx \SI{9.42426e-3}{\km\tothe{\frac{2}{3}}%
                                   \hour\tothe{\frac{1}{4}}},\\
      a_4 &= \SI{8.0}{\m\cubed\s\tothe{-\frac{1}{2}}}
          &&= (\num{8.0})(\SI{e-3}{\km})^3
                        (\SI[quotient-mode=fraction]{1/3600}{\hour})^{-\frac{1}{2}}
          &&= \SI{4.80000e-7}{\km\cubed\hour\tothe{-\frac{1}{2}}},
    \end{alignat*}
    %
    hvor en selvsagt også bare kunne ha regnet ut måltallet da potensene til enhetene
    ikke endres. Det nye \bquote{måltallet} til henholdsvis $a$ og $b$ blir nå
    %
    \begin{align*}
        \bpar{a} 
        &= \bpar{a_1}^{91}\bpar{a_3}^{96} \bpar{a_4}^{-82} \\
        & = (\num{2.25491e-3})^{91} (\num{9.42426e-3})^{96} (\num{4.8e-7})^{-82} \\
        &= \num{6.31995e82}, \\
        \bpar{b} 
        &= \bpar{a_2}^{5}\bpar{a_3}^{\frac{42}{13}} \bpar{a_4}^{-\frac{67}{39}} \\
        &= (\num{7.28711e-1})^{5}(\num{9.42426e-3})^{\frac{42}{13}} (\num{4.8e-7})^{-\frac{67}{39}} \\
        &= \num{4.20088e3}.
    \end{align*}
    %
    Vi ser at vi får ca de samme tallene som før. Grunnen til at $a$ avviker såvidt
    er at vi her arbeider med svært høye potenser. Ett lite avik i $\text{tall}^{92}$ 
    gjør mye større utsalg på det endelige svaret enn $\text{tall}^3$. Dersom vi skulle ha fått 
    enda likere svar måtte vi ha brukt enda flere desimaler. Direkte utregning av enhetene gir
    \begin{align*}
         \brak{a} 
      &= \brak{a_1}^{91}%
         \brak{a_3}^{96}%
         \brak{a_4}^{-82},\\
      &= (\si{\m\squared\s\tothe{{-\frac{5}{7}}}})^{91}
         (\si{\m\tothe{\frac{2}{3}}\s\tothe{\frac{1}{4}}})^{96}
         (\si{\m\cubed\s\tothe{-\frac{1}{2}}})^{-82} \\
      &= \si{\m\tothe{(2 \cdot 91 + \frac{2}{3}\cdot 96 - 3\cdot 82)}%
             \s\tothe{(-\frac{5}{7} \cdot 91 + \frac{1}{4}\cdot 96 + \frac{1}{2}\cdot 82)}}\\
      &= \si{\m\tothe{0}\s\tothe{0}} = 1,\\
         %
         \brak{b} 
      &= \brak{a_2}^{5}%
         \brak{a_3}^{\frac{42}{13}}%
         \brak{a_4}^{-\frac{67}{39}} \\
      &= (\si{\m\tothe{\frac{3}{5}}\s\tothe{-\frac{1}{3}}})^{5}
         (\si{\m\tothe{\frac{2}{3}}\s\tothe{\frac{1}{4}}})^{\frac{42}{13}}
         (\si{\m\cubed\s\tothe{-\frac{1}{2}}})^{\frac{67}{39}} \\
      &= \si{\m\tothe{(\frac{3}{5} \cdot 5 + \frac{2}{3} \cdot \frac{42}{13} + 3 \cdot \frac{67}{39})}%
             \s\tothe{(-\frac{1}{3} \cdot 5 + \frac{1}{4}\cdot \frac{42}{13} + \frac{1}{2}\cdot \frac{67}{39})}}\\
      &= \si{\m\tothe{0}\s\tothe{0}} = 1.\\
    \end{align*}
    %
    Slik at $a$ og $b$ er ubenevnte avledede størrelser, som var det som skulle vises.
\end{solution}

%==============================================================================%
%                           NOTAT 2.6.6 | PROBLEM 8                            %
%==============================================================================%

\begin{problem}
    \begin{subproblem}
        \label{subproblem:MAT-0001-Problem-N-2.2.6.a}
        La $M$, $R$, $T$ og
        $G$ være benevnte størrelser hvor benevningene er
        gitt ved
        \begin{align*}
          \bpar{M} &= \si{\kg},\\
          \bpar{R} &= \si{\m},\\
          \bpar{T} &= \si{\s},\\
          \bpar{G} &= \si{\m\cubed\per\kg\per\s\squared}.
        \end{align*}
        Vis at
        \begin{equation*}
          \alpha = T G^{\frac{1}{2}} M^{\frac{1}{2}} R^{-\frac{3}{2}}
        \end{equation*}
        er en ubenevnt avledet størrelse.
    \end{subproblem}
\end{problem}
\begin{solution}
    Direkte innsetning gir at
    %
    \begin{align*}
        \alpha 
        & = T G^{\frac{1}{2}} M^{\frac{1}{2}} R^{-\frac{3}{2}} \\
        & = (\si{\s})(\si{\m\cubed\per\kg\per\s\squared})^{\frac{1}{2}} (\si{\kg})^{\frac{1}{2}} (\si{\m})^{-\frac{3}{2}} \\
        & = \si{\s\tothe{1 - 2\cdot\frac{1}{2}}%
                \m\tothe{3 \cdot \frac{1}{2} - \frac{3}{2}}%
                \kg\tothe{-1\cdot\frac{1}{2}+\frac{1}{2}}} \\
        & = \si{\s\tothe{0}\m\tothe{0}\kg\tothe{0}}
          = 1.
    \end{align*}
    %
    som var det som skulle vises.
\end{solution}
    \begin{subproblem}
        La størrelsene fra \cref{subproblem:MAT-0001-Problem-N-2.2.6.a} ha måltall
        \begin{align*}
          \brak{M} &= \num{1.3},\\
          \brak{R} &= \num{10.4},\\
          \brak{T} &= \num{7.8},\\
          \brak{G} &= \num{23.7}.
        \end{align*}
        Bruk dette til å regne ut måltallet for størrelsen $\alpha$ fra \cref{subproblem:MAT-0001-Problem-N-2.2.6.a},
        og sjekk resultatet du fant der ved å vise at dersom vi endrer enheter
        \begin{align*}
          \si{\m}&\rightarrow \si{\milli\m},\\
          \si{\s}&\rightarrow \si{\hour},\\
          \si{\kg}&\rightarrow \si{\lbs}.
        \end{align*}
        så endres ikke måltallet for størrelsen $\alpha$.
    \end{subproblem}
\begin{solution}
    Vi begynner med å regne ut måltallet for størrelsen $\alpha$
    %
    \begin{align*}
           \brak{\alpha}
        &= \brak{T} \brak{G}^{\frac{1}{2}} \brak{M}^{\frac{1}{2}} \brak{R}^{-\frac{3}{2}} \\
        &= (\num{7.8}) (\num{23.7})^{\frac{1}{2}} (\num{1.3})^{\frac{1}{2}} (\num{10.4})^{-\frac{3}{2}} \\
        &\approx \num{1.13219}.
    \end{align*}
    %
    Dersom vi nå endrer enheter får vi i stedet
    %
        \begin{align*}
          M &= \SI{1.3}{\kg} 
             = \num{1.3}\cdot(\SI{2.205}{\lbs}) 
             = \SI{2.8665}{\lbs}\\
          R &= \SI{10.4}{\m} 
             = \num{10.4}\cdot(\SI{e3}{\milli\m}) 
             = \SI{1.04e4}{\milli\m}\\
          T &= \SI{7.8}{\s} 
             = \num{7.8} \cdot \biggl(\SI[quotient-mode=fraction]{1/3600}{\hour}\biggr)
             \approx \SI{2.16e-3}{\hour},\\
          G &= \SI{23.7}{\m\cubed\per\kg\per\s\squared} \\
            &= \num{23.7}(\SI{e3}{\milli\m})^3
                         (\SI{2.205}{\lbs})^{-1}
                   \biggl(\SI[quotient-mode=fraction]{1/3600}{\hour}\biggr)^{-2} \\
            &\approx \SI{1.392980e17}{\milli\m\cubed\per\lbs\per\hour\squared}
        \end{align*}
    %
    Vi regner så ut måltallet igjen med disse nye enhetene
    %
    \begin{align*}
        \brak{\alpha}
        &= \brak{T} \brak{G}^{\frac{1}{2}} \brak{m}^{\frac{1}{2}} \brak{r}^{-\frac{3}{2}} \\
        &= (\num{2.16e-3}) (\num{1.39298e17})^{\frac{1}{2}} (\num{2.8665})^{\frac{1}{2}} (\num{1.04e4})^{-\frac{3}{2}} \\
        &\approx \num{1.13219},
    \end{align*}
    %
    som er det samme som vi fikk tidligere.
\end{solution}
    \begin{subproblem}
        La de benevnte størrelsene $M$, $R$, $T$ fra \cref{subproblem:MAT-0001-Problem-N-2.2.6.a} være
        massen, rotasjonsperioden og radiusen til en geostasjonær bane for en
        planet. Legg merke til at den geostasjonære banen til en planet er en
        sirkulær bane hvor rotasjontiden for en satellitt er lik
        rotasjonsperioden for planeten. Sett fra bakken vil derfor satellitten
        alltid befinne seg i samme posisjon på himmelen.  Dette er opplagt
        nyttig dersom satellitten for eksempel sender TV-signaler til kunder på
        bakken.

        Den benevnte størrelsen $G$ er en universell konstant på lik linje med
        lysets hastighet i vakuum og kalles for \emph{gravitasjonskonstanten}.
        Benevningen er slik som gitt i \cref{subproblem:MAT-0001-Problem-N-2.2.6.a} og måltallet er gitt ved
        \begin{equation*}
          \brak{G} = \SI{6.67408e-11}.
        \end{equation*}
        Beregn verdien av den ubenevnte avledede størrelsen $\alpha$ for jorda
        og Mars. Måltallene for massen, rotasjonsperioden og radiusen for
        geostasjonær bane for jorda og Mars finner dere på nettet.

        Legg merke tillat det er mulig måltallene for størrelsene $M$, $R$ og
        $T$ dere finner på nettet ikke referer til enhetene $\si{\kg}$,
        $\si{\m}$ og $\si{\s}$. Dersom dette er tilfelle må dere skifte enheter
        for $m$, $r$ og $T$ til $\si{\kg}$, $\si{\m}$ og $\si{\s}$, noe som
        medfører at måltallene endrer seg ifølge den generelle regelen beskrevet
        på slutten av avsnitt (\textcolor{\UiTnumbercolor}{1}.\textcolor{\UiTnumbercolor}{7}.\textcolor{\UiTnumbercolor}{1}). Alternativtkan dere skifte
        enhetene $\si{\kg}$, $\si{\m}$ og $\si{\s}$ til enhetene som måltallene
        dere finner på nettet refererer til. Da vil måltallet til
        Gravitasjonskonstanten $G$ endre seg slik som beskrevet på slutten av
        avsnitt (\textcolor{\UiTnumbercolor}{1}.\textcolor{\UiTnumbercolor}{7}.\textcolor{\UiTnumbercolor}{1}).
    \end{subproblem}
    
\begin{solution}
    Vi ønsker å beregne $\alpha = T G^{\frac{1}{2}} M_{\text{j}}^{\frac{1}{2}} R^{-\frac{3}{2}}$
    hvor $T$ er omløptstiden til den geostasjonære satellitten, denne er gitt som ett
    siderisk døgn\footnote{se \url{https://nn.wikipedia.org/wiki/Stjerned\%C3\%B8gn} for definisjonen.}, $G$ er gravtiasjonskonstanten, $M_{\text{j}}$ er massen til jorden
    og $R$ er radiusen.
    
    \begin{center}
        \begin{tabular}{r c c c c}
            \toprule 
             Planet & {{$T$ ($\si{\s}$)}} & {{$G$ ($\si{\m\cubed\per\kg\per\s\squared}$)}} 
                    & {$M$ ($\si{\kg}$)} & {$R$ ($\si{\m}$)} \\ \midrule
             Jorden &   \num{86164} & \num{6.67430e-11} & \num{5.972e24} & \num{3.579e7} \\
             Mars   &   \num{88642} & \num{6.67430e-11} & \num{6.417e23} & \num{1.703e7} \\
             \bottomrule 
        \end{tabular}
    \end{center}
    
    Dette gjør at vi kan regne ut $\alpha_{\text{j}}$ og $\alpha_{\text{m}}$ for 
    henholdsvis jorden og mars
    %
    \begin{align*}
              \alpha_{\text{j}}
        &= T_{\text{j}} G^{\frac{1}{2}} M_{\text{j}}^{\frac{1}{2}} R_{\text{j}}^{-\frac{3}{2}} \\
        &=    \num{86164} 
        \cdot (\num{6.67430e-11})^{\frac{1}{2}}
        \cdot (\num{5.972e24})^{\frac{1}{2}} 
        \cdot (\num{3.579e7})^{-\frac{3}{2}} \\
        &\approx \num{8.03426},\\
              \alpha_{\text{m}}
        &= T_{\text{m}} G^{\frac{1}{2}} M_{\text{m}}^{\frac{1}{2}} R_{\text{m}}^{-\frac{3}{2}} \\
        &=    \num{88642} 
        \cdot (\num{6.67430e-11})^{\frac{1}{2}}
        \cdot (\num{6.417e23})^{\frac{1}{2}} 
        \cdot (\num{1.703e7})^{-\frac{3}{2}} \\
        &\approx \num{8.02366}.
    \end{align*}
    %
    Slik at vi ser at $\alpha$ er omtrent konstant.
\end{solution}

%==============================================================================%
%                           NOTAT 2.6.6 | PROBLEM 10                           %
%==============================================================================%

\begin{problem}[10]
    \begin{subproblem}
      \label{subproblem:MAT-0001-Problem-N-2.2.10.a}
      La $v$, $l$ og $g$ være benevnte størrelser hvor benevningene er
        gitt ved
        \begin{align*}
          \bpar{v} &= \si{\m\per\s},\\
          \bpar{l} &= \si{\m},\\
          \bpar{g} &= \si{\m\per\s\squared}.
        \end{align*}
        Vis at $\alpha = v l^{-\frac{1}{2}} g^{-\frac{1}{2}}$ er en ubenevnt avledet størrelse.
    \end{subproblem}
\end{problem}
\begin{solution}
    Ved direkte innsetning ser vi at 
    %
    \begin{align*}
            \alpha
        &=  v l^{-\frac{1}{2}} g^{-\frac{1}{2}} \\
        &= (\si{\m\per\s})(\si{\m})^{-\frac{1}{2}}(\si{\m\per\s\squared})^{-\frac{1}{2}}, \\
        &= \si{\m\tothe{1 -\frac{1}{2} - \frac{1}{2}}\s\tothe{-1-2\cdot(-\frac{1}{2})}} \\
        &= \si{\m\tothe{0}\s\tothe{0}} = 1,
    \end{align*}
    %
    slik at $\alpha = v l^{-\frac{1}{2}} g^{-\frac{1}{2}}$ er en ubenevnt avledet størrelse.
\end{solution}
    \begin{subproblem}
      La størrelsene fra \cref{subproblem:MAT-0001-Problem-N-2.2.10.a} ha måltall
        \begin{align*}
          \brak{v} &= \num{3.4},\\
          \brak{l} &= \num{1.2},\\
          \brak{g} &= \num{4.5}.
        \end{align*}
        Regn ut måltallet for størrelsen $\alpha$ fra \cref{subproblem:MAT-0001-Problem-N-2.2.10.a} og sjekk resultatet
        du fant der ved å vise at dersom vi endrer enheter
        \begin{align*}
          \si{\m} &\rightarrow \si{\ft},\\
          \si{\s} &\rightarrow \si{\hour}.
        \end{align*}
        så endres ikke måltallet for størrelsen $\alpha$.
    \end{subproblem}
    \iftoggle{isLF}{%
        La oss først regne ut måltallet til $\alpha$
        %
        \begin{equation*}
            \brak{a} 
            = \brak{v} \brak{l}^{-\frac{1}{2}} \brak{g}^{-\frac{1}{2}}
            = \num{3.4} (\num{1.2})^{-\frac{1}{2}} (\num{4.5})^{-\frac{1}{2}}
            \approx \num{1.46313}.
        \end{equation*}%
        %
        Ved å endre enheter får vi derimot
        %
        \begin{align*}
            v &= \SI{3.4}{\m\per\s}
               = \num{3.4}\cdot
                 \SI{3.281}{\ft}
                 \biggr(\SI[quotient-mode=fraction]{1/3600}{\hour}\biggl)^{-1}\\
              &\approx \SI{4.015944e4}{\ft\per\hour}, \\
            l &= \SI{1.2}{\m} = \num{1.2}\cdot \SI{3.281}{\ft}  \\
              &= \SI{3.9372}{\ft},\\
            g &= \SI{4.5}{\m\per\s\squared}
               = \num{4.5}\cdot
                 \SI{3.281}{\ft}
                 \biggr(\SI[quotient-mode=fraction]{1/3600}{\hour}\biggl)^{-2}\\
              &= \SI{1.91347920e8}{\ft\per\hour\squared}.
        \end{align*}
        %
        Dermed får vi at måltallet til $\alpha$ med nye enheter kan skrives
        %
        \begin{align*}
               \brak{a} 
            &= \brak{v} \brak{l}^{-\frac{1}{2}} \brak{g}^{-\frac{1}{2}} \\
            &= (\num{40159.44}) 
               (\num{3.9372})^{-\frac{1}{2}} 
               (\num{1.91347920e8})^{-\frac{1}{2}}
            \approx \num{1.46313},
        \end{align*}%
        %
        som var det samme resultatet som tidligere.
    }{}
    \begin{subproblem}
      La oss betrakte en situasjon hvor en person befinner seg på en
        planet hvor tyngdens akselerasjon er $g$. Tyngdens akselerasjon på en
        planet er akselerasjonen et legeme vil oppleve dersom det slippes og får
        falle fritt nær planetens overflate. La $l$ være lengden på bena til
        personen og la $v$ være farten personen oppnår ved å gå på planetens
        overflate på den mest energieffektive måten. På jorda tyder målinger
        gjort i laboratorier at $\alpha=1$. Bruk dette til å beregne din egen
        mest energieffektive gangfart på jorda. La oss postulerer at samme verdi
        for $\alpha$ gjelder for alle planeter. Hva er din egen mest
        energieffektive gangfart på Månen og på Mars. Du finner tyngdens
        akselerasjon for disse himmellegemene på nettet.
    \end{subproblem}
    \iftoggle{isLF}{%
        Her må vi først løse likningen med tanke på $v$
        %
        \begin{align*}
            \alpha &= v l^{-\frac{1}{2}} g^{-\frac{1}{2}} \\
            v &= \frac{\alpha}{l^{-\frac{1}{2}} g^{-\frac{1}{2}}}
              = l^{\frac{1}{2}} g^{\frac{1}{2}}.
        \end{align*}
        %
        Vi kan nå sette opp en tabell for å regne ut $v$ på ulike planeter
        
        \begin{center}
            \begin{tabular}{r c c c}
                \toprule 
                 Planet & $l$ ($\si{\m}$) & $g$ ($\si{\m\per\s\squared}$) & {$v = l^{\frac{1}{2}}g^{\frac{1}{2}}$ ($\si{\m\per\s})$} \\
                 \midrule
                 Jorden & \num{1.0} & \num{9.807} & \num{3.132} \\
                  Månen & \num{1.0} & \num{1.620} & \num{1.273} \\ 
                   Mars & \num{1.0} & \num{3.711} & \num{1.926}\\ 
              \bottomrule
            \end{tabular}
        \end{center}
        
        Altså er min optimale gangfart på jorden ca lik $\SI{3.1}{\m\per\s}$.
    }{}
\end{document}
