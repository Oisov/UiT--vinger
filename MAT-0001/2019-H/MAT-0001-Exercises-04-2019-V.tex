\documentclass[a4paper,11pt]{article}

% Always keep your main language last
\usepackage[british,nynorsk,samin,norsk]{babel}

\usepackage{_sty/UiT}
\usepackage{_sty/IMS}
\usepackage{MAT-0001/MAT-0001}

%==============================================================================%
%                              COURSE INFORMATION                              %
%==============================================================================%


\UiTsetup{%
    Language = {norsk}, %auto = language defined by babel
    isExercise = {true}, % Is this an exercise? true / false
    ExerciseNumber = {4}, % Which exercise number is this?
    Solution = {true}, % Boolean (true/false) includes solutions
    SolutionShort = {auto}, % is used by \UiTcorrect and \UiTwrong
    CourseCode = {auto}, % auto = \courseCode (defined in courseCode.sty, e.g. MAT-1001.sty)
    CourseName = {auto}, % auto = \courseName (defined in courseCode.sty, e.g. MAT-1001.sty)
    Year = {2019}, % No auto setting for this, as it might change on recompilation
    Month = {9}, % No auto setting for this, as it might change on recompilation
    Day = {16},% No auto setting for this, as it might change on recompilation
    DurationDays = {4}, % Set the number of days the exam/exercise should last
}

%==============================================================================%
%                     OWN COMMANDS AND PACKAGES BELOW HERE                     %
%==============================================================================%

\DeclareDocumentCommand{\drawQuadratic}{O{0} O{0} m}{%
    \pgfmathsetmacro{\minX}{ 1.50 + #1}
    \pgfmathsetmacro{\maxX}{ 8.25 + #1}
    \pgfmathsetmacro{\maxY}{ 1.25 + #2}
    \pgfmathsetmacro{\minY}{-9.25 + #2}
\begin{tikzpicture}[declare function={
            f(\x)= #3;
          }] 
            \begin{axis}[%
                    UiTplotstyle,
                    width=2*\textwidth,
                    ytick={-9,-8,...,9},
                    xtick={-3,-2,...,9},
                    xmin= \minX, xmax=\maxX,
                    ymin= \minY, ymax=\maxY,
                    domain = 2:8,
                    axis x discontinuity=crunch,
               after end axis/.code={
                  \path (axis cs:0,0) 
                  node [anchor=west,xshift=0.16cm] {0};
                }]
       \addplot[ultra thick, UiT-blue, smooth] {f(x)}; 
       \addplot+[samples=7, only marks,mark=*,thick,UiT-blue,mark options = {scale=1.5,fill=white}] {f(x)}; 
    \end{axis}
\end{tikzpicture} 
}

\begin{document}

%==============================================================================%
%                                 EXERCISE 04                                  %
%==============================================================================%

\frontpageUiT

\titlebox[norsk]{auto}{\Spraak}{auto}{\dateDuration}

%==============================================================================%
%                                SEKSJON 2.1                                   %
%==============================================================================%

\UNIsection[2]

%==============================================================================%
%                               PROBLEM 2.1.2                                  %
%==============================================================================%

\begin{problem}[2]
    La $f$ være funksjonen definert ved $f(x) = 2x - 3$, med definisjonsområde
    $D_f = [-1, 2)$. Tegn en skisse av grafen til $f$.
\end{problem}

\begin{solution}
    Løsningen er vist i \cref{fig:MAT-0001-Problem-2-1-2}, men hvordan kommer vi dit? Det enkleste er nok først å bare regne ut noen verdier, som vist i
    \cref{tab:MAT-0001-Problem-2-1-2} deretter kan vi markere disse punktene på grafen, også trekke en linje mellom dem. I utgangspunktet trenger vi bare $2$ punkter for å tegne en rett linje, men her tok vi med $4$.
\end{solution}

\hvisLF{%
    \medskip
    \hspace{-3cm}
    \begin{minipage}{\textwidth}
        \centering
        \begin{tikzpicture}
          \begin{axis}[
            UiTplotstyle,
            ytick={-5,-4,...,1},
            xtick={-1,0,1,2},
            % x tick label style={below right},
            % y tick label style={above left,},
            ymin=-5.5,
            ymax= 1.5,
            xmin=-1.5,
            xmax= 2.5,
            legend style={at={(0.25,0.145)},anchor=west},
            legend cell align={left},
            ]
            \addplot+[domain=-1:2, mark=none, ultra thick, UiT-blue] {2*x - 3};
            \legend{$f(x) = 2x - 3$}
            \addplot+[domain=-1:2, mark=*, only marks, samples=4, mark options={ultra thick, color=UiT-blue, fill=white, scale=1.5}] {2*x - 3};
          \end{axis}
        \end{tikzpicture}
        \captionof{figure}{}
        \label{fig:MAT-0001-Problem-2-1-2}
    \end{minipage}%
    \hspace{-7cm}
    \begin{minipage}{\textwidth}
        \centering
        \captionof{table}{}
        \begin{tabular}{c|r r r r}
          \toprule
               $x$  & $-1$ &  $0$ &  $1$ & $\phantom{-}2$ \\
          \midrule
             $f(x)$ & $-5$ & $-3$ & $-1$ & $1$ \\ 
          \bottomrule
        \end{tabular}
        \label{tab:MAT-0001-Problem-2-1-2}
    \end{minipage}%
}

%==============================================================================%
%                               PROBLEM 2.1.3                                  %
%==============================================================================%

\begin{problem}[3]
    Tegn grafen til funksjonen $g$ bestemt ved at $D_g = (-\infty, 3]$ og
    %
    \begin{equation*}
        g(x) = \begin{cases}
            x  - 2             & \text{hvis} \phantom{1 <.} x \neq -2, \\
            \phantom{x-}\ \, 5 & \text{hvis} \phantom{1 <.} x = -2, \\
            x                  & \text{hvis} \ 1 < x \leq 3.
        \end{cases}
    \end{equation*}
\end{problem}

\begin{solution}
    Akkurat som før begynner vi med å sette opp en tabell med $x$ og $y$ verdier. 
    Det viktige her er at vi må ha ett området som er stort nok til å få med alle 
    bitene av funksjonen. Når $x \leq -2$ så er $g(x) = x - 2$ og tilsvarende når
    $x > 3$ så er også $g(x) = x - 2$. Slik at dersom $x \in [-3, 4]$ får vi med
    hele oppførselen til $g$. De tilhørende verdiene er vist i 
    \cref{tab:MAT-0001-Problem-2-1-3}
    
    \begin{center}
        \captionof{table}{}
        \label{tab:MAT-0001-Problem-2-1-3}
        \begin{tabular}{h | j j j j j j j j }
            \toprule
               x  & -3 & -2 & -1 & 0 & 1 & \phantom{-}2 & \phantom{-}3 & \phantom{-}4  \\
            \midrule
             f(x) & -5 & 5 & -5 & -2 & -1 & 2 & 3 & -1 \\ 
            \bottomrule
        \end{tabular}
    \end{center}
    
    Dersom vi tegner en strek mellom disse punktene $(-1, -3), (0,-2), \ldots$
    får vi \cref{fig:MAT-0001-Problem-2-1-2}. Merk, det er viktig å passe på hva
    verdien er i skjøtepunktene. Eksempelvis så er $g(1) = -1$ selv om $g(\num{1.0001}) = \num{1.0001}$
    fordi $g(x) = x - 2$ når $x \in (-2,1]$, men $g(x) = x$ når $x \in (1,3)$.
\end{solution}

\hvisLF{%
    \begin{figure}[htbp!]
        \centering
    \begin{tikzpicture}[
          declare function={
            func(\x)= (\x < -2) * (\x - 2) +
                      (\x == -2) * (5)   +
                      and(\x > -2, \x <= 1) * (\x - 2)     +
                      and(\x > 1, \x <= 3) * (\x)     +
                      (\x > 3) * (\x - 5)
           ;
          }
        ]
          \begin{axis}[
            UiTplotstyle,
            ytick={-5,-4,...,5},
            xtick={-3,-2,...,4},
            % x tick label style={below right},
            % y tick label style={above left,},
            ymin=-5.5,
            ymax= 5.5,
            xmin=-3.5,
            xmax= 4.5,
            legend style={at={(0.25,0.145)},anchor=west},
            legend cell align={left},
            ]
            \addplot+[domain=-3:4, samples=200, mark=none, ultra thick, UiT-blue] {func(x)};
            \addplot+[domain=-3:4, mark=*, only marks, samples=8, mark options={ultra thick, color=UiT-blue, fill=white, scale=1.5}] {func(x)};
          \end{axis}
        \end{tikzpicture}
        \caption{}
        \label{fig:MAT-0001-Problem-2-1-2}
    \end{figure}
}

%==============================================================================%
%                               PROBLEM 2.1.5                                  %
%==============================================================================%

\begin{problem}[5]
    Undersøk hvilke av følgende utsagn som definerer $y$ som en funksjon av $x$
\end{problem}

\begin{subproblem}
    $2x - 2y = 5$ \UiTcorrect
\end{subproblem}

\begin{solution}
    En enkel måte er å se om det er mulig å løse likningen med hensyn på $y$.
    Altså om det for enhver $x$ eksisterer en \emph{unik} tilhørende $y$ verdi.
    %
    \begin{equation}
        2x - 2y = 5
        \ \rightarrow \ 
        2y = 5 - 2x
        \ \rightarrow \
        y = \frac{5}{2} - x.
    \end{equation}
    %
    Altså definerer uttrykket $2x - 2y = 5$, $y$ som en funksjon av $x$.
\end{solution}

\begin{subproblem}
    $x^2 + y^2 = 1$ \UiTwrong
\end{subproblem}

\begin{solution}
    Vi ser igjen om det er mulig å løse likningen med hensyn på $y$
    %
    \begin{equation}
        x^2 + y^2 = 1
        \ \rightarrow \ 
        y^2 = 1 - x^2
        \ \rightarrow \
        y = \pm \sqrt{1 - x^2}.
    \end{equation}
    %
    Hvor vi ser at dersom $x \neq 0$ så har hver $y$ verdi \emph{to} tilhørende $x$
    verdier.
    Altså definerer \emph{ikke} uttrykket $2x - 2y = 5$, $y$ som en unik funksjon av $x$.
\end{solution}

\begin{subproblem}
    $y$ er overflaten av en kule med volum $x$ \UiTwrong
\end{subproblem}

\begin{solution}
    Arealet av en sfære er $y=4\pi r^2$ så først må vi finne radiusen
    \begin{align*} 
        \text{Volum:} \ x &= \frac{4}{3}\pi r^3 \ \Rightarrow \ r = \sqrt[3]{\frac{3x}{4\pi}} \\
        \text{Areal:} \ y &= 4 \pi r^2 = 4 \pi \biggl(\sqrt[3]{\frac{3x}{4\pi}}\biggr)^{2}= 6^{\frac{2}{3}}\pi^{\frac{1}{3}} x^{\frac{2}{3}}
    \end{align*}
    Slik at vi ser at for ethvert volum $x$ så eksisterer det ett unikt tilhørende
    areal $y$. Intuitivt gir dette mening, da en kule med volum $x$ ikke kan ha
    flere ulike overflateareal.
\end{solution}

\begin{subproblem}
    $y$ er omkretsen av et rektangel med areal $x$ \UiTcorrect
\end{subproblem}

\begin{solution}
    Akkurat som i forrige oppgave må vi tenke om det eksisterer rektangler som
    har samme areal, men ulike omkretser. Jeg er sikker på at du klarer å komme
    på noen eksempler om du tenker litt. Slik at utsagnet ikke stemmer, men la 
    oss for ordens skyld sette det opp formelt. La rektangelet ha lengde $l$ og
    bredde $x$
    \begin{align*} 
        \text{areal:} \   x = lb \qquad \text{og} \qquad
        \text{omkrets:} \ y = 2l + 2b = 2(1 + \frac{x}{l}) 
    \end{align*}
    Dersom for eksempel arealet er $x = 12$ så ser vi at vi kan variere
    lengden av rektangelet $l$ og få ulike verdier for omkretsen.
    Se \cref{fig:MAT-0001-Problem-2-1-5} 
    for eksempler på rektangler med samme areal, men ulik omkrets.
    
    \begin{center}
        \begin{tikzpicture}[rectangle style/.style={thick,UiT-blue},scale=0.75]
        \draw[step=1.0,gray,thin,dashed] (0,0) grid (12,8);
        \draw[rectangle style] (0,0) rectangle ++(12,1);
        \draw[rectangle style] (0,2) rectangle ++(2,6);
        \draw[rectangle style] (3,2) rectangle ++(3,4);
        \draw[rectangle style] (7,2) rectangle ++(3.46410,3.46410);
        \end{tikzpicture}
        \captionof{figure}{}
        \label{fig:MAT-0001-Problem-2-1-5}
    \end{center}
\end{solution}

%==============================================================================%
%                               PROBLEM 2.1.7                                  %
%==============================================================================%

\begin{problem}[7]
    Skisser grafen til $f(x) = 2x^2 + 4x + 1$.
\end{problem}

\begin{solution}
    En metode er som før og plotte masse punkter, men en smartere måte 
    kan være å gjøre
    %
    \begin{itemize}
        \item Ekstremalpunkt: (Topp-,bunnpunkt)
        \item Nullpunkt: Hvor $f$ skjærer $x$-aksen
        \item Hvor $f$ skjærer $y$-aksen
        \item Tegne fortengslinje for $f$
        \item Tegne fortegnslinje for $f'$
    \end{itemize}
    %
    Da vil en få en mye mer naturlig fremstilling av funksjonen enn hva
    punktene vil gi oss. 
    %
    \begin{align*}
        2x^2 + 4x + 1 = 0
         \ \Rightarrow \ x = \frac{-4 \pm \sqrt{4^2 - 4\cdot 2}}{2 \cdot 2} = -1 \pm \frac{\sqrt{2}}{2} = -1 \pm \frac{1}{\sqrt{2}} 
    \end{align*}
    %
    Hvor vi nå definerer $x_{-} = -1 - 1/2\sqrt{2}$ og $x_{+} = -1 + 1/\sqrt{2}$.
    Anta nå at vi ikke har en kalkulator tilgjengelig, hvordan skal vi finne ut 
    hvor punktene $(x_{-},0)$ og $(x_{+}, 0)$ befinner seg? Vel, siden 
    $1 < \sqrt{2} < 2$ får vi følgende ulikheter
    %
    \begin{align*}
          -2 &= -1 - \frac{1}{1} < -1 - \frac{1}{\sqrt{2}} < -1 - \frac{1}{2} = - \frac{3}{2} \\
-\frac{1}{2} &= -1 + \frac{1}{2} < -1 + \frac{1}{\sqrt{2}} < -1 + \frac{1}{1} = 0
    \end{align*}
    %
    Slik at $-2 < x_{-} < -\num{1.5}$ og $\num{-0.5}<x_{+}<0$, som er mer enn nøyaktig nok for en skisse.
    Mens den deriverte er $f'(x) = 4x + 4$ slik at $f'(x) = 0 \Rightarrow x = -1$.
    Vi er nå klar til å drøfte funksjonen i en fortegnslinje
    
    \begin{center}
        \begin{functionallines}[
            extra x tick labels={$x_{-}$,$x_{+}$},
            xtick={-2, -1, 0, 1, 2}]{-1.7071,-0.29289}{-2.5:1}
            \functionalline[x - x_{-}]{x+1.7071}{1}{-1.7071}
            \functionalline[x - x_{+}]{x+0.29289}{2}{-0.29289}
            \functionalline[f(x)]{(x+1.7071)*(x+0.29289)}{3}{-1.7071,-0.29289}
            \functionalline[f'(x)]{4*x + 4}{4}{-1}
        \end{functionallines}
    \end{center}
    
    Hvor vi ser at funksjonen synker frem til $x_{-}$, så krysser den $x$-aksen
    og fortsetter å synke til den når bunnpunktet $(-1,-1)$. Deretter stiger den
    til den når $x_{+}$ også fortsette den å stige. Grafen til $f(x)$ er vist i
    \cref{fig:MAT-0001-Problem-2-1-2} hvor vi plasserte $x_{-}$ ca midt mellom
    $-2$ og $-1.5$ og $x_{+}$ ca midt mellom $-\num{0.5}$ og $0$.
\end{solution}

\hvisLF{%
    \begin{figure}[htbp!]
        \centering
    \begin{tikzpicture}
          \begin{axis}[
            UiTplotstyle,
            ytick={-5,-4,...,7},
            xtick={-3,-2,...,4},
            % x tick label style={below right},
            % y tick label style={above left,},
            ymin=-1.5,
            ymax= 4.25,
            xmin=-2.75,
            xmax= 1.25,
            domain = -2.5:0.5,
       after end axis/.code={
          \path (axis cs:0,0) 
          node [anchor=north west,yshift=-0.075cm] {0};
        }
          ]
            \addplot[smooth, mark=none, ultra thick, UiT-blue] {2*x*x + 4*x + 1};
            \addplot+[only marks,mark=*,thick,UiT-blue,mark options = {scale=1.5,fill=white}] plot coordinates
                { (-2,1) (-1,-1) (0,1)};
            \addplot+[only marks,mark=*,thick,UiT-orange,mark options = {scale=1.5,fill=white}] plot coordinates
                { (-1.7071,0) (-0.29289,0)};
          \end{axis}
        \end{tikzpicture}
        \caption{}
        \label{fig:MAT-0001-Problem-2-1-2}
    \end{figure}
}

\newpageNotLF

%==============================================================================%
%                                SEKSJON 2.2                                   %
%==============================================================================%

\UNIsection

%==============================================================================%
%                               PROBLEM 2.2.1                                  %
%==============================================================================%

\begin{problem}[1]
    La $f$ være funksjonen gitt ved
    %
    \begin{equation}
        \label{eq:MAT-0001-Problem-2-2-1}
        f(x) = 3 + \frac{1}{x}, \qquad D_f = (0, \infty).
    \end{equation}
\end{problem}

\begin{subproblem}
    Hva skjer med $f(x)$ når $x$ nærmer seg $0$?
\end{subproblem}

\begin{solution}
    Siden definisjonsområdet er $(0, \infty)$ så kan vi bare
    nærme oss $0$ fra høyre $\leftarrow$. Et kjent triks er at vi kan la
    $x = 1/n$ hvor $n$ er ett naturlig tall. At $x \to 0^+$ er det samme som at 
    $n \to \infty$. Hvor vi indikerte at vi beveger oss fra 
    den positive siden mot null ved å bruke $0^{+}$. Innsetning gir
    %
    \begin{equation*}
        f(x) = f(\frac{1}{n}) = 3 + \frac{1}{(1/n)} = 3 + n
    \end{equation*}
    %
    Hvor det er veldig enkelt å se at $f(x) \to \infty$ når $n \to \infty$
    som vi husker var det samme som at $x \to 0^{+}$.
\end{solution}

\begin{subproblem}
    Vis at $f$ er injektiv. Finn $V_f$.
\end{subproblem}

\begin{solution}
    Begrepet injektiv kan være litt vanskelig, men det betyr at hver eneste funksjonsverdi $y$ har maksimalt én tilhørende x-verdi. En måte å vise dette på er å vise at \cref{eq:MAT-0001-Problem-2-2-1} bare synker eller bare vokser\footnote{Fjonge matematikkere kaller gjerne dette for monotont voksende eller monotont synkende funksjoner.}.
    Om du tenker litt etter så er det ikke mulig at funksjonsverdien $y$ har flere tilhørende
    $x$-verdier dersom $y$ er voksende eller synkende. 
    %
    \begin{equation}
        f'(x) = -\frac{1}{x^2}
    \end{equation}
    %
    Siden $x \in (0, \infty)$ er $f'(x) < 0$ for alle $x \in D_f$ slik at $f$ er monotont synkende. 
    Dette betyr igjen at funksjonen vår er injektiv! Mens verdimengden er alle mulige $x$-verdier
    vi kan ha, er verdimengden alle mulige $y$-verdier funksjonen vår kan oppnå. 
    Siden $f(x) \to \infty$ når $x \to 0^{0}$ og $f(x) \to 3$ når $x \to \infty$ så blir
    $V_f = (3,\infty)$. Kan kan altså nå alle $y$-verdier høyere enn $3$, som gir veldig mye mening 
    når du tegner grafen.
\end{solution}

\begin{subproblem}
    Finn den inverse funksjonen $f^{-1}$ til $f$. Angi definisjonsmengde for $f^{-1}$.
\end{subproblem}

\begin{solution}
    En måte å finne inversen på er å løse \cref{eq:MAT-0001-Problem-2-2-1} med hensyn på $x$
    %
    \begin{equation}
        f(x) = 3 + \frac{1}{x} \ \Rightarrow \ 
        x = \frac{1}{f(x) - 3} \ \Rightarrow \ 
        f^{-1}(x) = \frac{1}{x - 3}
    \end{equation}
    %
    Hvor vi ser at for hver $y = f(x)$ så eksisterer det en unik tilhørende $x$ verdi, gitt ved
    formelen ovenfor. Den enkleste måten er å si at
    %
    \begin{equation}
        D_f \Leftrightarrow V_{f^{-1}} 
        \quad \text{og} \quad 
        V_f \Leftrightarrow D_{f^{-1}} 
    \end{equation}
    %
    Men vi kan og se dette fra likningen $f^{-1}(x) = 1/(x - 3)$. Nå er $f^{-1}(x)$ alle
    $x$-verdiene funksjonen vår kan ta, siden $x$ er positiv må også $1/(x - 3)$ være det.
    Dette betyr at definisjonsmengden til $f^{-1}$ er $D_{f^{-1}} = (3, \infty)$ som er det
    samme som $V_f$.
\end{solution}

\begin{subproblem}
    Hva skjer med $f^{-1}(x)$ når $x$ blir stor?    
\end{subproblem}

\begin{solution}
    For store $x$ har vi at $f^{-1} = 1/(x -3) \to 0$ når $x \to \infty$.
    Som igjen gir mening for den inverse funksjonen har jo vi bare byttet om 
    på $x$ og $y$. Siden $f \to \infty$ når $x \to 0^+$ må vi nødvendigvis ha
    at $f^{-1} \to 0$ når $x \to \infty$.
\end{solution}

\begin{subproblem}
    Tegn grafene til $f$ og $f^{-1}$ på samme figur.
\end{subproblem}

\begin{solution}
    Vi har tegnet $f$ og $f^{-1}$ i \cref{fig:MAT-0001-Problem-2-2-1-e}.
    Det som kan være viktig å merke seg er at $f^{-1}$ hele tiden er 
    speilet omkring linjen $y = x$. Dette er jo fordi 
    $f(x) = y \Leftrightarrow f^{-1}(y) = x$ så for hvert punkt $(x^*, y^*)$
    som ligger på grafen til $f$ så ligger ett tilsvarende punkt $(y^*, x^*)$
    på grafen til $f^{-1}$.
\end{solution}

\hvisLF{%
    \begin{figure}[htbp!]
        \centering
    \begin{tikzpicture}
          \begin{axis}[
            UiTplotstyle,
            ytick={-5,-4,...,7},
            xtick={-3,-2,...,5},
            % x tick label style={below right},
            % y tick label style={above left,},
            ymin= 0,
            ymax= 5.25,
            xmin=-0.25,
            xmax= 5.25,
            domain = 0.01:5,
       after end axis/.code={
          \path (axis cs:0,0) 
          node [anchor=north west,yshift=-0.075cm] {0};
        }
          ]
            \addplot[mark=none, ultra thick, UiT-blue] {3 + 1/x};
            \addplot[domain=3.1:5, mark=none, ultra thick, UiT-orange] {1/(x - 3)};
            \addplot[domain=0:5, mark=none, dashed, ultra thick, black] {x};
          \end{axis}
        \end{tikzpicture}
        \caption{}
        \label{fig:MAT-0001-Problem-2-2-1-e}
    \end{figure}
}

\begin{subproblem}
    Vis at
    \begin{subsubproblem*}
        \item $f^{-1}\bigl(f(x)\bigr) = x \ \forall\footnote{Symbolet $\forall$ er en \bquote{universal quantification} og leses som \emph{for alle} på Norsk.} \ x \in D_f$,
        \label{subsubproblem:MAT-0001-Problem-2-2-1-e-i}
        \item $f\bigl(f^{-1}(x)\bigr) = x \ \forall \ x \in V_f$.
        \label{subsubproblem:MAT-0001-Problem-2-2-1-e-ii}
    \end{subsubproblem*}
\end{subproblem}

\begin{solution}
    \Opg{subsubproblem:MAT-0001-Problem-2-2-1-e-i} Rett frem regning gir
    %
    \begin{equation}
          f^{-1}\bigl(f(x)\bigr)
        = \frac{1}{f(x) - 3}
        = \frac{1}{(3 + \frac{1}{x}) - 3}
        = \frac{1}{1/x} \frac{x}{x}
        = x
    \end{equation}
    \Opg{subsubproblem:MAT-0001-Problem-2-2-1-e-i} Helt tilsvarende så får vi
    %
    \begin{equation}
          f\bigl(f^{-1}(x)\bigr)
          = 3 + 1 \left/\biggl(\frac{1}{x - 3}\biggr)\right.
          = 3 + (x - 3)
          = x
    \end{equation}
\end{solution}

%==============================================================================%
%                                SEKSJON 2.3                                   %
%==============================================================================%

\UNIsection[2]

%==============================================================================%
%                               PROBLEM 2.3.1                                  %
%==============================================================================%

\begin{problem}[1]
    Angi følgende punktmengde ved et skavert område i $xy$-planet:
    %
    \begin{equation*}
        \Set{(x,y) \given x - y \geq 2}.
    \end{equation*}
\end{problem}

\begin{solution}
    Vi ønsker altså området slik at $y \leq x - 2$ ved å snu om på
    ulikheten $x - y \geq 2$.Med andre ord ønsker vi området 
    under eller på linjen $y = x - 2$. Dette området er vist i \cref{fig:MAT-0001-Problem-2-3-1-e}
\end{solution}

\hvisLF{%
    \begin{figure}[htbp!]
        \centering
        \begin{tikzpicture} 
            \begin{axis}[%
                    UiTplotstyle,
                    ytick={-5,-4,...,7},
                    xtick={-3,-2,...,5},
                    % x tick label style={below right},
                    % y tick label style={above left,},
                    ymin= -3.5,
                    ymax=  3.5,
                    xmin= -1.5,
                    xmax=  5.5,
                    domain=-2:6,
               after end axis/.code={
                  \path (axis cs:0,0) 
                  node [anchor=north west,yshift=-0.075cm] {0};
                }]
            \addplot[no markers, draw=none,name path=f] {x - 2};
            \addplot+[no markers, draw=none,name path=axis] {-3.5};
        
            \addplot [
                thick,
                color=UiT-blue,
                fill=UiT-blue, 
                fill opacity=0.2
            ]
            fill between[
                of=f and axis,
                soft clip={domain=-3:6},
            ];
            \end{axis}
        \end{tikzpicture} 
        \caption{}
        \label{fig:MAT-0001-Problem-2-3-1-e}
    \end{figure}
}

%==============================================================================%
%                               PROBLEM 2.3.2                                  %
%==============================================================================%

\begin{problem}[2]
    Følgende uliketer bestemmer et trapesformet området i $xy$-planet
    %
    \begin{equation}
        x \geq 0, \qquad 
        y \geq 0, \qquad 
        x + y \geq 2, \qquad 
        x + y \leq 4.
    \end{equation}
    %
    Skaver området, og bestem områdets hjørner.
\end{problem}

\begin{solution}
    Det enkleste på en slik oppgave er å først bare tegne linjene uten ulikheter
    %
    \begin{equation*}
        x = 0, \qquad 
        y = 0, \qquad 
        x + y = 2, \qquad 
        x + y = 4.
    \end{equation*}
    %
    Som vist i \cref{fig:MAT-0001-Problem-2-3-2} det er nå enkelt å se hvilket området
    som ulikhetene avgrenser. Hjørnene er $\Set{(2,0), (4,0), (0,4), (0,2)}$.
\end{solution}

\hvisLF{%
    \begin{figure}[htbp!]
        \centering
        \begin{tikzpicture} 
            \begin{axis}[%
                    UiTplotstyle,
                    ytick={-5,-4,...,7},
                    xtick={-3,-2,...,5},
                    xmin=-1, xmax=5,
                    ymin=-1, ymax=5,
                    legend cell align={left},
               after end axis/.code={
                  \path (axis cs:0,0) 
                  node [anchor=north west,yshift=-0.075cm] {0};
                }]
       \addplot[samples=100, ultra thick, UiT-blue,domain=-1:3, name path=A] {2 - x)}; 
      \addplot[samples=50, ultra thick, UiT-orange, domain=-1:5,name path=B] {4 - x}; 
      \path[name path=xaxis] (\pgfkeysvalueof{/pgfplots/xmin}, 0) -- (\pgfkeysvalueof{/pgfplots/xmax},0);
       \addplot[color=UiT-main, pattern=north west lines,forget plot] fill between[of=B and xaxis, soft clip={domain=2:4}];
      \addplot[color=UiT-main, pattern=north west lines,forget plot] fill between[of=A and B, soft clip={domain=0:2}];
      \addplot +[mark=none,ultra thick, color=UiT-green] coordinates {(0, -1) (0, 5)};
      \addplot +[mark=none,ultra thick, color=UiT-red] coordinates {(-1, 0) (5, 0)};
      \legend{$y = 2-x$, $y = 4-x$, $x = 0$, $y = 0$}
        \addplot+[only marks,mark=*,thick,UiT-blue,mark options = {scale=1.5,fill=white}] plot coordinates
                { (2,0) (4,0) (0,4) (0,2)};
    \end{axis}
        \end{tikzpicture} 
        \caption{}
        \label{fig:MAT-0001-Problem-2-3-2}
    \end{figure}
}

%==============================================================================%
%                               PROBLEM 2.3.3                                  %
%==============================================================================%

\begin{problem}[3]
    Følgende uliketer bestemmer et trapesformet området i $xy$-planet
    %
    \begin{equation}
        x \geq 0, \qquad 
        y \geq 0, \qquad 
        x + y \leq 2.
    \end{equation}
\end{problem}

\begin{solution}
    Igjen ved å tegne linjene
    %
    \begin{equation}
        x = 0, \qquad y = 0, \qquad x + y = 2
    \end{equation}
    %
    får vi \cref{fig:MAT-0001-Problem-2-3-3}.
\end{solution}

\hvisLF{%
    \begin{figure}[htbp!]
        \centering
        \begin{tikzpicture} 
            \begin{axis}[%
                    UiTplotstyle,
                    ytick={-5,-4,...,7},
                    xtick={-3,-2,...,5},
                    xmin=-1.25, xmax=3.25,
                    ymin=-1.25, ymax=3.25,
                    legend cell align={left},
               after end axis/.code={
                  \path (axis cs:0,0) 
                  node [anchor=north west,yshift=-0.075cm] {0};
                }]
       \addplot[samples=100, ultra thick, UiT-blue,domain=-1:3, name path=A] {2 - x)}; 
      \path[name path=xaxis] (\pgfkeysvalueof{/pgfplots/xmin}, 0) -- (\pgfkeysvalueof{/pgfplots/xmax},0);
       \addplot[color=UiT-main, pattern=north west lines,forget plot] fill between[of=A and xaxis, soft clip={domain=0:2}];
      \addplot +[mark=none,ultra thick, color=UiT-green] coordinates {(0, -1.25) (0, 3.25)};
      \addplot +[mark=none,ultra thick, color=UiT-red] coordinates {(-1.25, 0) (3.25, 0)};
      \legend{$y = 2-x$, $x = 0$, $y = 0$}
        \addplot+[only marks,mark=*,thick,UiT-blue,mark options = {scale=1.5,fill=white}] plot coordinates
                {(2,0) (0,0) (0,2)};
       \addplot[black,domain=-1:3] {-1 - 2*x)}; 
       \addplot[black,domain=-1:3] { 0 - 2*x)}; 
       \addplot[black,domain=-1:3] { 1 - 2*x)};
       \addplot[black,domain=-1:3] { 2 - 2*x)};
       \addplot[black,domain=-1:3] { 3 - 2*x)}; 
       \addplot[black,domain=-1:3] { 4 - 2*x)}; 
    \end{axis}
        \end{tikzpicture} 
        \caption{}
        \label{fig:MAT-0001-Problem-2-3-3}
    \end{figure}
}

\begin{subproblem}
    \label{subproblem:MAT-0001-Problem-2-3-3-a}
    Skaver området, og bestem områdets hjørner.
\end{subproblem}

\begin{solution}
    Igjen så er det skaverte området vist i \cref{fig:MAT-0001-Problem-2-3-3},
    mens hjørnene er gitt som $\Set{(2,0) (0,0) (0,2)}$.
\end{solution}

\begin{subproblem}
    Skisser, på samme figur som i \cref{subproblem:MAT-0001-Problem-2-3-3-a}, 
    noen nivålinjer for funksjonen
    %
    \begin{equation*}
        f(x, y) = 2x + y.
    \end{equation*}
\end{subproblem}

\begin{subproblem}
    Hvor i området har $f$ sin største og minste verdi?
\end{subproblem}

\begin{solution}
    Her må vi se hvilke av nivålinjene som går igjennom hjørnene til området vårt.
    Vi ser at det er tre nivålinjer som gjør dette
    %
    \begin{align*}
        & 0 = 2x + y \ \text{treffer} \ \vek{x}_1 = (0,0), \ f(\vek{x}_1) = 2 \cdot 0 + 0 = 0 \\
        & 2 = 2x + y \ \text{treffer} \ \vek{x}_2 = (0,2), \ f(\vek{x}_2) = 2 \cdot 0 + 2 = 2 \\
        & 4 = 2x + y \ \text{treffer} \ \vek{x}_3 = (2,0), \ f(\vek{x}_3) = 2 \cdot 2 + 0 = 4
    \end{align*}
    %
    \begin{itemize}
        \item Funksjonen $f$ oppnår sin \emph{maksimale verdi} i punktet 
          $\vek{x}_{\text{maks}} = (2,0)$ med tilhørende verdi 
          $f_{\text{maks}} = f(\vek{x}_{\text{maks}}) = 4$.
        \item Funksjonen $f(x)$ oppnår sin \emph{minimale verdi} i punktet 
          $\vek{x}_{\text{min}} = (0,0)$ med tilhørende verdi 
          $f_{\text{min}} = f(\vek{x}_{\text{min}}) = 0$.
    \end{itemize}
\end{solution}

%==============================================================================%
%                                SEKSJON 2.X                                   %
%==============================================================================%

\UNIsection*

%==============================================================================%
%                               PROBLEM 2.X.3                                  %
%==============================================================================%

\begin{problem}[3]
    \textit{Flytting av grafer}. I denne oppgaven skal vi studere hvilken effekt 
    det har på grafen til en funksjon at vi endrer funksjonsuttrykket på noen 
    spesielle måter. Det er meningen du skal skissere på frihånd, uten 
    kalkulator. Kvalitativ forståelse er poenget her.
\end{problem}

\begin{subproblem}
    Skisser grafen til $f(x) = x^2$.
\end{subproblem}

\begin{solution}
    Merk at funksjonen er symmetrisk omkring $y$-aksen, slik at vi bare
    trenger å finne punkter til høyre for $y$-aksen, også speile verdiene.
    Videre så vil funksjonen alltid kvadrere det vi legger inn. Dette gir
    følgende verditabell
    
    \begin{center}
        \begin{tabular}{j | *{4}h}
            \toprule
               x & 0 & 1 & 2 & 3 \\
            \midrule
            f(x) & 0 & 1 & 4 & 9 \\
            \bottomrule
        \end{tabular}
    \end{center}
\end{solution}

\hvisLF{%
    \begin{figure}[htbp!]
        \centering
        \begin{tikzpicture} 
            \begin{axis}[%
                    UiTplotstyle,
                    ytick={-1,0,...,9},
                    xtick={-3,-2,...,5},
                    xmin=-3.25, xmax=3.25,
                    ymin=-0.25, ymax=9.25,
                    domain = -3:3,
               after end axis/.code={
                  \path (axis cs:0,0) 
                  node [anchor=north west,yshift=-0.075cm] {0};
                }]
       \addplot[ultra thick, UiT-blue, smooth] {\x*\x)}; 
       \addplot+[only marks,mark=*,thick,UiT-blue,mark options = {scale=1.5,fill=white}] plot coordinates
                {(-3,9) (-2,4) (-1,1) (0,0) (1,1) (2,4) (3,9)};
    \end{axis}
        \end{tikzpicture} 
        \caption{}
        \label{fig:MAT-0001-Problem-2-3-3}
    \end{figure}
}

\begin{subproblem}
    Skisser grafen til $f(x) = x^2 + 3$. Hvilken effekt hadde det på grafen 
    at vi la til $3$ i funksjonsuttrykket?
\end{subproblem}

\begin{solution}
    Igjen så kan vi gjøre akkurat det samme som sist. Vi lager en verditabell, også plotter vi punktene fra verditabellen. Igjen så er funksjonen vår symmetrisk så vi plotter bare punktene til høyre for $x$-aksen.
    
    \begin{center}
        \begin{tabular}{j | *{4}h}
            \toprule
               x & 0 & 1 & 2 & 3 \\
            \midrule
            f(x) & 3 & 4 & 7 & 12 \\
            \bottomrule
        \end{tabular}
    \end{center}
    
\end{solution}

\hvisLF{%
    \begin{figure}[htbp!]
        \centering
        \begin{tikzpicture} 
            \begin{axis}[%
                    UiTplotstyle,
                    ytick={-1,0,...,9},
                    xtick={-3,-2,...,5},
                    xmin=-3.25, xmax=3.25,
                    ymin=-0.25, ymax=9.25,
                    domain = -3:3,
               after end axis/.code={
                  \path (axis cs:0,0) 
                  node [anchor=north west,yshift=-0.075cm] {0};
                }]
       \addplot[ultra thick, UiT-blue, smooth] {\x*\x + 3)}; 
       \addplot+[only marks,mark=*,thick,UiT-blue,mark options = {scale=1.5,fill=white}] plot coordinates
                {(-3,12) (-2,7) (-1,4) (0,3) (1,4) (2,7) (3,12)};
    \end{axis}
        \end{tikzpicture} 
        \caption{}
        \label{fig:MAT-0001-Problem-2-3-3}
    \end{figure}
}

\begin{subproblem}
    Skisser grafen til $f(x) = x^2 - 4$.
\end{subproblem}

\begin{solution}
    Igjen så kan vi gjøre akkurat det samme som sist. Vi lager en verditabell, også plotter vi punktene fra verditabellen. Igjen så er funksjonen vår symmetrisk så vi plotter bare punktene til høyre for $x$-aksen.
    
    \begin{center}
        \begin{tabular}{j | *{4}h}
            \toprule
               x & 0 & 1 & 2 & 3 \\
            \midrule
            f(x) & -4 & -3 & 0 & 5 \\
            \bottomrule
        \end{tabular}
    \end{center}
    
\end{solution}

\hvisLF{%
    \begin{figure}[htbp!]
        \centering
        \begin{tikzpicture} 
            \begin{axis}[%
                    UiTplotstyle,
                    ytick={-4,-3,...,9},
                    xtick={-3,-2,...,5},
                    xmin=-3.25, xmax=3.25,
                    ymin=-4.25, ymax=5.25,
                    domain = -3:3,
               after end axis/.code={
                  \path (axis cs:0,0) 
                  node [anchor=north west,yshift=-0.075cm] {0};
                }]
       \addplot[ultra thick, UiT-blue, smooth] {\x*\x - 4)}; 
       \addplot+[only marks,mark=*,thick,UiT-blue,mark options = {scale=1.5,fill=white}] plot coordinates
                {(-3,5) (-2,0) (-1,-3) (0,-4) (1,-3) (2,0) (3,5)};
    \end{axis}
        \end{tikzpicture} 
        \caption{}
        \label{fig:MAT-0001-Problem-2-3-3}
    \end{figure}
}

\begin{subproblem}
    Skisser grafen til $f(x) = 2x^2$. Hvilken effekt hadde det at vi ganget 
    med $2$?
\end{subproblem}

\hvisLF{%
    \begin{figure}[htbp!]
        \centering
        \begin{tikzpicture} 
            \begin{axis}[%
                    UiTplotstyle,
                    ytick={-1,0,...,9},
                    xtick={-3,-2,...,5},
                    xmin=-3.25, xmax=3.25,
                    ymin=-0.25, ymax=9.25,
                    domain = -3:3,
               after end axis/.code={
                  \path (axis cs:0,0) 
                  node [anchor=north west,yshift=-0.075cm] {0};
                }]
       \addplot[ultra thick, UiT-blue, smooth] {2*\x*\x)}; 
       \addplot+[only marks,mark=*,thick,UiT-blue,mark options = {scale=1.5,fill=white}] plot coordinates
                {(-3,18) (-2,8) (-1,2) (0,0) (1,2) (2,8) (3,18)};
    \end{axis}
        \end{tikzpicture} 
        \caption{}
        \label{fig:MAT-0001-Problem-2-3-3}
    \end{figure}
}

\begin{solution}
    Siden vi nå har $f(x) = 2x^2$ så øker $f$ \emph{dobbelt} så fort som $x^2$.
    Vi kan dermed anta at funksjonen har lik form, men er \emph{smalere}
    da den raskere når høyere verdier. 
    
    \begin{center}
        \begin{tabular}{j | *{4}h}
            \toprule
               x & 0 & 1 & 2 & 3 \\
            \midrule
            f(x) & 0 & 2 & 8 & 18 \\
            \bottomrule
        \end{tabular}
    \end{center}
    
\end{solution}

\begin{subproblem}
    Skisser grafen til $f(x) = -x^2$. Hvilken effekt hadde det på grafen at 
    vi satte minustegn foran funksjonsuttrykket?
\end{subproblem}

\begin{solution}
    Ved litt tenking ser vi at dette speiler funksjonen omkring $x$-aksen.
    Som gir følgende verditabell
    
    \begin{center}
        \begin{tabular}{j | *{4}h}
            \toprule
               x & 0 &  1 &  2 &  3 \\
            \midrule
            f(x) & 0 & -1 & -4 & -9 \\
            \bottomrule
        \end{tabular}
    \end{center}
\end{solution}

\hvisLF{%
    \begin{figure}[htbp!]
        \centering
        \begin{tikzpicture} 
            \begin{axis}[%
                    UiTplotstyle,
                    ytick={-9,-8,...,1},
                    xtick={-3,-2,...,5},
                    xmin=-3.25, xmax=3.25,
                    ymin= -9.25, ymax=0.25,
                    domain = -3:3,
               after end axis/.code={
                  \path (axis cs:0,0) 
                  node [anchor=north west,yshift=-0.075cm] {0};
                }]
       \addplot[ultra thick, UiT-blue, smooth] {-\x*\x)}; 
       \addplot+[only marks,mark=*,thick,UiT-blue,mark options = {scale=1.5,fill=white}] plot coordinates
                {(-3,-9) (-2,-4) (-1,-1) (0,0) (1,-1) (2,-4) (3,-9)};
    \end{axis}
        \end{tikzpicture} 
        \caption{}
        \label{fig:MAT-0001-Problem-2-3-3}
    \end{figure}
}

\begin{subproblem}
    Skisser grafen til $f(x) = (x - 5)^2$. Hint: regn ut $f(5)$.
    Hvilken effekt hadde det på grafen at vi erstattet $x$ med $(x - 5)$?
\end{subproblem}

\begin{subproblem}
    Skisser grafene til $f(x) = -(x-5)^2$, $g(x)=-2(x-5)^2$, $h(x) = 4 - (x-5)^2$.
\end{subproblem}

\begin{figure}
    \centering
    \begin{subfigure}[b]{0.3\textwidth}
        \centering
        \drawQuadratic{-(\x-5)^2} 
        \caption{$f(x) = -(x - 5)^2$}
        \label{subfig:MAT-0001-Problem-2-3-3-e-i}
    \end{subfigure}
    \hfill %add desired spacing between images, e. g. ~, \quad, \qquad, \hfill etc. 
      %(or a blank line to force the subfigure onto a new line)
    \begin{subfigure}[b]{0.3\textwidth}
        \centering
        \drawQuadratic{-2*(\x-5)^2} 
        \caption{$g(x) = -2(x - 5)^2$}
        \label{subfig:MAT-0001-Problem-2-3-3-e-ii}
    \end{subfigure}
    \hfill %add desired spacing between images, e. g. ~, \quad, \qquad, \hfill etc. 
    %(or a blank line to force the subfigure onto a new line)
    \begin{subfigure}[b]{0.3\textwidth}
        \centering
        \drawQuadratic[0][4]{4-(\x-5)^2} 
        \caption{$h(x) = 4-(x - 5)^2$}
        \label{subfig:MAT-0001-Problem-2-3-3-e-iii}
    \end{subfigure}
    \caption{}\label{fig:MAT-0001-Problem-2-3-3-e}
\end{figure}

%==============================================================================%
%                               PROBLEM 2.X.4                                  %
%==============================================================================%

\begin{problem}[4]
    \label{problem:MAT-0001-Problem-2-X-4}
    Finn et uttrykk for den omvendte funksjonen til $f$:
\end{problem}

\begin{subproblem}{3}
    \item $f(x) = 2 - 3x$
    \label{subproblem:MAT-0001-Problem-2-X-4-a}
    \item $g(x) = 1 + x^3$
    \label{subproblem:MAT-0001-Problem-2-X-4-b}
    \item \vspace*{-0.8cm}$h(x) = \cfrac{1}{x - 1}$
    \label{subproblem:MAT-0001-Problem-2-X-4-c}
\end{subproblem}

\begin{solution}
    \def\flushspace{\hspace{2cm}}
    \begin{alignat*}{6}
           &\ref{subproblem:MAT-0001-Problem-2-X-4-a} \flushspace
           f(x) &&= 2 - 3x 
           &&\Rightarrow x 
           &&= \frac{2 - f(x)}{3} 
           &&\Rightarrow f^{-1}(x) 
           &&= \frac{2 - x}{3} \flushspace \\
         &\ref{subproblem:MAT-0001-Problem-2-X-4-b} \flushspace 
           g(x) &&= 1 + x^3 
           &&\Rightarrow x 
           &&= \sqrt[3]{g(x) - 1} 
           &&\Rightarrow g^{-1}(x) 
           &&= \sqrt[3]{x - 1} \flushspace \\
         &\ref{subproblem:MAT-0001-Problem-2-X-4-c} \flushspace
           h(x) &&= \frac{1}{x - 1} 
           &&\Rightarrow x 
           &&= \frac{1}{h(x)} + 1  
           &&\Rightarrow h^{-1}(x) 
           &&= \frac{1}{x} + 1 \flushspace 
    \end{alignat*}
\end{solution}

%==============================================================================%
%                               PROBLEM 2.X.5                                  %
%==============================================================================%

\begin{problem}[5]
    For hver av funksjonene i \cref{problem:MAT-0001-Problem-2-X-4}, vis ved 
    utregning at for alle $a$ og $b$ har
    %
    \begin{equation*}
        f^{-1}\bigl(f(a)\bigr) = a,
        \qquad \text{og} \qquad 
        f\bigl(f^{-1}(b)\bigr) = b.
    \end{equation*}
\end{problem}

\newpageNotLF

%==============================================================================%
%                                NOTAT 2.6.6                                   %
%==============================================================================%

\UNInotat[2]{6}{6}

\begin{align}
  \langle&a\rangle
  = \brak{a_1}^{p_1}\brak{a_2}^{p_2}\dots\brak{a_n}^{p_n},
  \label{Mregel}\\
  [&a]
  = \,\bpar{a_1}^{p_1}\,\bpar{a_2}^{p_2}\,\dots\,\bpar{a_n}^{p_n}
  \label{Bregel}.
\end{align}

Vi har altså at for enhver benevnt avledet
størrelse $a$, så finnes det potenser $q_1,\dots,q_k$ slik at
%
\begin{equation}
  \bpar{a} = d_1^{q_1}\cdots d_k^{q_k},\label{BasicUnitExpansion}
\end{equation}
%
hvor $d_1,\cdots,d_k$ er kortnavnene for enhetene assosierte med de benevnte
grunnstørrelsene.

%==============================================================================%
%                           NOTAT 2.6.6 | PROBLEM 3                            %
%==============================================================================%

\begin{problem}[3]
    La følgene benevnte størrelser være gitt
    \begin{align*}
      a_1 &= \SI{3.30}{\m\tothe{\frac{1}{2}}%
                                \s\cubed%
                                \kg\tothe{\frac{1}{3}}},\\
      a_2 &= \SI{4.50}{\m\tothe{-\frac{2}{3}}%
                                \s\tothe{\frac{1}{2}}%
                                \kg\tothe{\frac{1}{2}}},\\
      a_3 &= \SI{6.71}{\m\squared%
                                 \s\tothe{-\frac{1}{3}}\kg\cubed},
    \end{align*}
    og la $a$ være den benevnte avledede størrelsen
    \begin{equation*}
      a = a_1^2 a_2^{\tfrac{1}{2}} a_3^{\tfrac{1}{5}},
    \end{equation*}
\end{problem}
\vspace{-0.5cm}
\begin{subproblem}
    Finn måltallet til $a$ ved å benytte
    regelen \eqref{Mregel} for å finne måltallet til en benevnt avledet
    størrelse.
\end{subproblem}
\vspace{-0.25cm}
\begin{subproblem}
    Vis, ved å benytte regelen \eqref{Bregel} for å finne
    benevningen til en benevnt avledet størrelse, at benevningen til
    $a$ kan skrives på formen \eqref{BasicUnitExpansion}. Angi hva
    $k$, $d_1,\dots,d_k$ og $p_1,\dots, p_k$ er i dette tilfellet.
\end{subproblem}
\vspace{-0.25cm}
\begin{subproblem}
    Skift fra gamle enheter til nye enheter ved at
    \begin{equation*}
      \si{\kg} \rightarrow \si{\ct}, \qquad
      \si{\s} \rightarrow \si{\hour}, \qquad
      \si{\m} \rightarrow \si{\ft}.
    \end{equation*}
    Beregn måltall og benevning til $a$ med hensyn på de nye
    enhetene på to måter
    \begin{subsubproblem}
        \label{subsubproblem:MAT-0001-Exercise-03-Problem-N-2.6.6.1-c-i}
        Skift først enheter i $a_1$ og $a_2$ og finn de nye måltallene
        og benevningene tilhørende disse to benevnte størrelsene. Finn så
        nytt måltall og benevning til $a$ ved å bruke reglene fra \cref{Mregel,Bregel}.
    \end{subsubproblem}
    \begin{subsubproblem}
        \label{subsubproblem:MAT-0001-Exercise-03-Problem-N-2.6.6.1-c-ii}
        Bruk måltallet og benevningen til $a$ med hensyn på gamle
        enheter, funnet i oppgavene a og b, til å direkte beregne måltallet
        og benevningen til $a$ med hensyn på de nye enhetene.
    \end{subsubproblem}
\end{subproblem}

%==============================================================================%
%                          NOTAT 2.6.6 | PROBLEM 11                            %
%==============================================================================%

\begin{problem}[11]
    \begin{subproblem}
        La $E$, $T$, $\rho$ og
        $r$ være benevnte størrelser hvor benevningene er gitt ved
        \begin{align*}
          \bpar{E}    &= \si{\kg\m\squared\per\s\squared},\\
          \bpar{T}    &= \si{\s}\\
          \bpar{\rho} &= \si{\kg\per\m\cubed}\\
          \bpar{R}    &= \si{\m}.
        \end{align*}
        Vis at
        \begin{equation*}
          \alpha = E\,T^2 R^{-5} \rho^{-1}
        \end{equation*}
        er ubenevnt avledet størrelser.
    \end{subproblem}
\end{problem}
\begin{subproblem}
      La størrelsene fra a) ha måltall
        \begin{align*}
          \brak{E} &= \num{3.4},\\
          \brak{T} &= \num{5.1},\\
          \brak{\rho} &= \num{1.6},\\
          \brak{R} &= \num{2.5}.
        \end{align*}
        Regn ut måltallet for størrelsen $\alpha$ fra a)og sjekk resultatet
        du fant der ved å vise at dersom vi endrer enheter
        \begin{align*}
          \si{\m}&\rightarrow \si{\mm},\\
          \si{\s}&\rightarrow \si{\hour},\\
          \si{\kg}&\rightarrow g.
        \end{align*}
        så endres ikke måltallet for størrelsen $\alpha$.
    \end{subproblem}
    \begin{subproblem}
        La oss betrakte en situasjon hvor en eksplosjon har funnet sted for
        kort tid siden. La $r$ være radiusen til eksplosjonsskyen en tid $T$
        etter at eksplosjonen fant sted og la $\rho$ være lufttettheten utenfor
        eksplosjonsskyenen og $E$ energien som ble frigjort i eksplosjonen.
        Eksperimenter med små eksplosjoner i laboratorium tyder på at den
        ubenevnte avledede størrelsen har numerisk verdi nær $1$. La oss derfor
        anta at $\alpha=1$ og la oss postulere at den samme verdien av $\alpha$
        gjelder for alle eksplosjoner, store som små. Dette postulatet gir oss
        en fysisk lov som uttrykker energien som utløser en eksplosjon ved hjelp
        av de benevnte størrelsene $T$, $\rho$ og $r$
        %
        \begin{equation*}
          E = R^{5} \rho\,t^{-2}
        \end{equation*}
        %
        Den første test eksplosjonen av en atombombe ble utført i staten New
        Mexico i USA i 1945. Måltallet for energien som ble frigjort ved
        eksplosjonen ble hemmeligholdt da en ikke ønsket at land som Tyskland og
        Sovjetunionen, som stod mot USA og de allierte i andre verdenskrig,
        skulle få noe hint som kunne hjelpe dem til å konstruere egne
        atombomber. Like etter ble imidlertid et bilde av eksplosjonskyen
        $\SI{0.006}{\s}$ etter eksplosjonen publisert i et populærvitenskapelig
        magasin. Bilde inkluderte en lengdeskala som gjorde det mulig å regne ut
        at på dette tidspunkt etter eksplosjonen var radiusen til
        eksplosjonsskyen ca. $\SI{80}{\m}$. Lufttettheten på eksplosjonsstedet
        var ca.  $\SI{1.269}{\kg\per\m\cubed}$. Beregn energien frigjort ved
        eksplosjonen.

        Benevningen for den benevnte avledede størrelsen $E$ er som
        nevnt i del a) av oppgaven $\si{\kg\m\squared\per\s\squared}$. Denne
        benevningen kalles Joule, med kortnavn $\si{\joule}$. For eksplosjoner
        som frigjør store energimengder benyttes en annen enhet for energi som
        kalles kiloton, med kortnavn $\si{\kilo\tonne}$. Regelen for å
        konvertere måltallet for energi når en skifter fra enheten Joule til
        kiloton finner en på nettet og er
        \begin{equation*}
          \si{\joule} \rightarrow \SI{2.3901e-13}{\kilo\tonne}.
        \end{equation*}
        Finn måltallet for den frigjorte energien dersom vi bruker kiloton som
        enhet. Detaljerte målinger og beregninger som ble gjort i 1945 viste at
        den frigjorte energien var ca $\SI{21}{\kilo\tonne}$. Denne verdien var
        nær nok verdien du nettopp, med utgangspunkt i åpne kilder, har beregnet
        til at personen som først gjorde disse beregningene i 1945 ble mistenkt
        for å være en forræder. Dette fordi han frigjorde hemmelig informasjon
        som Tyskland og Sovjetunionen kunne bruke til å utvikle sine egne
        atombomber. I realiteten var han bare en ekspert i bruk av ubenevnte
        avledede størrelser. Han var en britisk anvendt matematiker ved navn G.
        I. Taylor.
    \end{subproblem}
%==============================================================================%
%                          NOTAT 2.6.6 | PROBLEM 14                            %
%==============================================================================%

\begin{problem}[14]
    La følgende matematiske modell være gitt
    \begin{align}
      ax^2=b,\label{OvingsModell1}
    \end{align}
    hvor parametrene $a$ og $b$ er følgende benevnte
    størrelser
    \begin{align*}
      a &= \SI{2.34}{\kg\squared\per\m\cubed\s\tothe{\frac{1}{3}}},\\
      b &= \SI{1.27}{\kg\tothe{-\frac{1}{2}}\m\tothe{\frac{1}{2}}\per\s\cubed},
    \end{align*}
\end{problem}
    \begin{subproblem}
        \label{subproblem:MAT-0001-Notat-2.6.6-Problem-14a}
      Hvilken benevning må den ukjente $x$ ha for at \cref{OvingsModell1}
        skal være en akseptabel matematisk modell?
        \end{subproblem}
        \begin{subproblem}
        \label{subproblem:MAT-0001-Notat-2.6.6-Problem-14b}
      La $y$ være måltallet til den ukjent, $y=\brak{x}$. Skriv ned
        ligningen for $y$.
        \end{subproblem}
        \begin{subproblem}
        \label{subproblem:MAT-0001-Notat-2.6.6-Problem-14c}
      Finn alle løsningene til ligningen som desimaltall,hvor mange
        signifikante sifre er det naturlig å kreve her?
        \end{subproblem}
        \begin{subproblem}
        \label{subproblem:MAT-0001-Notat-2.6.6-Problem-14d}
      Skriv ned de tilsvarende benevnte størrelsene som er løsninger til
        den matematiske modellen fra \cref{OvingsModell1}.
        \end{subproblem}
        \begin{subproblem}
            \label{subproblem:MAT-0001-Notat-2.6.6-Problem-14e}
      Vi ønsker nå å skifte enheter fra $\si{\m}$, $\si{\s}$ og $\si{\kg}$
        til $\si{\milli\m}$,$\si{\min}$ og $\si{\ct}$
        \begin{align*}
          \si{\m} &\rightarrow \SI{e3}{\milli\m},\\
          \si{\s} &\rightarrow \SI[quotient-mode=fraction]{1/60}{\min},\\
          \si{\kg} &\rightarrow \SI{5e3}{\ct}.
        \end{align*}
        La oss beregne løsningen $x$ på to forskjellige måter
        \begin{subsubproblem}
            Finn måltall og benevning til de to benevnte størrelsene
            $a$ og $b$, og finn hvilkenbenevning den ukjente
            $x$ må ha for at \cref{OvingsModell1} skal være en akseptabel
            matematisk modell. Skriv deretter ned ligningen for måltallet,
            $y=\brak{x}$ til den ukjente $x$ og finn alleløsningene med det
            antall signifikante sifre det er naturlig å kreve. Bruk så disse
            løsningene til å skrive ned de benevnte størrelsene som er løsninger
            til den matematiske modellen fra \cref{OvingsModell1}.
        \end{subsubproblem}
        \begin{subsubproblem}
            Vi kjenner benevningen til den ukjente med hensyn på de
            opprinnelige enhetene fra \cref{subproblem:MAT-0001-Notat-2.6.6-Problem-14a}. Brukt dette sammen med
            resultatet fra \cref{subproblem:MAT-0001-Notat-2.6.6-Problem-14d} og formlene for skifte av enheter til å
            finne måltall og benevning til de løsningene til
            \cref{OvingsModell1} etter skifte av enheter.
        \end{subsubproblem}
    \end{subproblem}
\end{document}