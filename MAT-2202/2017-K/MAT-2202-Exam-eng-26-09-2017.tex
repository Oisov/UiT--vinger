\titlebox[english]

Please show work justifying your answers and conclusions.

All ten sub-problems count equally.

\vspace{-0.25cm}
\Problem
    
\begin{subproblem}
    \label{subproblem:MAT-2202-26-09-2017-problem-1a}
    For the function
    %
    \begin{equation}
        f(x_1, x_2) = x_1^4 - x_1 x_2 + x_2^4
    \end{equation}
    %
    find all three stationary points.
\end{subproblem}

\begin{subproblem}
    For the function $f(x_1,x_2)$ in \cref{subproblem:MAT-2202-26-09-2017-problem-1a}, use
    $\vx_0 = (1, 1)$ as the initial point for a gradient search with constant step size $\alpha = \num{0.1}$, to calculate the iterate $\vx_1$.
\end{subproblem}

\begin{subproblem}
    \label{subproblem:MAT-2202-26-09-2017-problem-1c}
    For the function $f(x_1, x_2)$ in \cref{subproblem:MAT-2202-26-09-2017-problem-1a}, use
    $\widetilde{x}_0 = (1,1)$ as the initial point for a 
    Newton search, to calculate the iterate $\widetilde{x}_1$. 
\end{subproblem}

\begin{subproblem}
    Is the Newton step in \cref{subproblem:MAT-2202-26-09-2017-problem-1c} a descent direction?
\end{subproblem}

\vspace{-0.25cm}
\Problem

\begin{subproblem}
    \label{subproblem:MAT-2202-26-09-2017-4a}
    The linear program
    %
    \begin{align*}
        \phantom{2}x_1 + 2 x_2 + 4x_3 \to &\min \\
        4x_1 + \phantom{2}x_2 + 2x_3 \geq{}&{}3 \\
        2x_1 + 4x_2 + \phantom{4}x_3 \geq{}&{}5 \\
        x_1, x_2,x_3 \geq{}&{}0 
    \end{align*}
    %
    is in canonical form. Write down its dual linear program, and solve the dual.
    (Find an optimal solution and the corresponding optimal value).
\end{subproblem}

\begin{subproblem}
    \label{subproblem:MAT-2202-26-09-2017-4b}
    Writing down the linear program in \cref{subproblem:MAT-2202-26-09-2017-4a} in standard form
    %
    \begin{alignat*}{2}
        \phantom{2}x_1 + 2 x_2 + 4x_3 && \to &\min \\
        4x_1 + \phantom{2}x_2 + 2x_3 &&{}- e_1 \geq & \ 3 \\
        2x_1 + 4x_2 + \phantom{4}x_3 &&{}- e_2 \geq & \ 5 \\
        x_1, x_2,x_3,e_1,e_2 &&  \geq & \ 0 
    \end{alignat*}
    %
    Check that $\vx = \mat{0&1&1&0&0}\tran$ is a basic feasible solution. Establish that it is not an optimal solution.
\end{subproblem}

\begin{subproblem}
    Perform a single step of the simplex method
    to obtain from the vector $\vx = \mat{0&1&1&0&0}\tran$ a new basic feasible solution
    of the LP problem in
    \cref{subproblem:MAT-2202-26-09-2017-4b}
    that improves the objective.
\end{subproblem}

\begin{subproblem}
    The figure below shows a road network from A to F,
    with four toll booths at B, C, D and E. To pass each booth
    it is necessary to pay the marked toll. The problem of
    finding a least cost path from A to F can be formulated
    as an LP model called a \emph{capacitated minimum-cost network
    flow problem.} Write down such a model for the tolls.
    %
    \begin{center}
        \begin{tikzpicture}[every node/.style={draw,circle}]
            \node (A) at (0,0) {$A$};
            \node[label=above:{$\$\,1$},above right= of A] (B) {$B$};
            \node[label=below:{$\$\,2$},below right= of A] (C) {$C$};
            \node[draw=none,right= of B] (Bempty) {};
            \node[label=above:{$\$\,2$},right= of Bempty] (D) {$D$};
            \node[draw=none,right= of C] (Cempty) {};
            \node[label=below:{$\$\,1$},right= of Cempty] (E) {$E$};
            \node[above right= of E] (F) {$F$};
            
            \path [draw,postaction={on each segment={mid arrow}}]
            (A) -- (B)%
            (A) -- (C)%
            (B) -- (D)%
            (B) -- (E)%
            (C) -- (E)%
            (D) -- (F)%
            (E) -- (F)%
            ;
        \end{tikzpicture}
        \parbox{\linewidth}{\captionof{figure}{}}
    \end{center}
\end{subproblem}


\Problem

\begin{subproblem}
    \label{subproblem:MAT-2202-26-09-2017-4a}
    Show that
    \begin{align*}
        x^2 + 2y^2 & \to \min,\\
        x - y & \leq 0, \\
        x^2 + y^2 - 1 & \leq 0,
    \end{align*}
    %
    is a convex optimisation problem.
\end{subproblem}

\begin{subproblem}
    For the optimisation problem in \cref{subproblem:MAT-2202-26-09-2017-4a} write down the
complete Karush-Kuhn-Tucker (also called KKT) conditions. Explain how we can be certain, for this optimisation
problem, that any optimal solution will satisfy the KKT
conditions.
\end{subproblem}