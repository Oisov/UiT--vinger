%%^^A%%  fontspec-doc-luatex.tex -- part of FONTSPEC <wspr.io/fontspec>

\documentclass[a4paper]{l3doc}
\usepackage{fontspec-doc-style}
\showexamplesfalse
\begin{document}

\part{\LuaTeX-only font features}
\label{sec:luatex-features}

\section{Different font technologies and shapers}
\label{sec:renderer-luatex}

\LuaTeX\ does not directly support any font rendering technologies out of the box, it
requires additional functionality to be added to properly support and control technologies such as OpenType.

Using the \feat{Renderer} feature, there are a number of options that \pkg{fontspec} can pass to the engine to control which font technology is being used.
Pre-2019, there were two options provided by \pkg{luaotfload} that generally did not require user intervention.

\begin{itemize}
\item \feat{Renderer} = \opt{Node} : the default `mode' for typesetting OpenType fonts.
\item \feat{Renderer} = \opt{Base} : a simplified mode useful only in a limited number of situations such as mathematics typesetting.
\end{itemize}

From 2019 the possibility of using the Harfbuzz text shaping engine within \LuaTeX\ has
been developed by Khaled Hosny. When running a suitable \LuaTeX\ engine with Harfbuzz support, \pkg{fontspec} provides the following options:

\begin{itemize}
\item \feat{Renderer} = \opt{Harfbuzz} : use the Harfbuzz engine without an explicit `shaper'.
\item \feat{Renderer} = \opt{OpenType} : use the Harfbuzz engine with the OpenType shaper.
\item \feat{Renderer} = \opt{AAT} : use the Harfbuzz engine with the AAT shaper.
\item \feat{Renderer} = \opt{Graphite} : use the Harfbuzz engine with the Graphite shaper.
\item \feat{Renderer} = \meta{foo} : use the Harfbuzz engine with the \meta{foo} shaper.
\end{itemize}

Support for the Harfbuzz renderer is preliminary and may be improved over time.
Please treat the interface for Harfbuzz fonts as subject to change.


\section{Custom font features}

\LuaTeX, via the \pkg{luaotfload} package, allows the definition and re-definition of custom OpenType features for a selected font. This facility is particularly useful to implement custom substitutions or to disable unwanted but not all ligatures.

Figure~\ref{fig:featurefile} shows an minimal example of this type of functionality.
This example creates a new OpenType feature, \texttt{oneb}, which substitutes the glyph
when typesetting `\texttt{1}' for the named glyph \texttt{one.ss01}. The glyph names
are font specific and can be interrogated with third-party software such as \emph{FontForge}.

A third-party collection of additional examples are maintained in the repository `\texttt{fonts-in-luatex}'\footnote{\url{https://github.com/mewtant/fonts-in-luatex}}.
These examples are intended to correct or adjust font features in a range of commercial fonts and provide a good introduction to some of the possibilities that \LuaTeX\ affords.

Please refer to the \LuaTeX/\pkg{luaotfload} documentation for more details.

\begin{figure}
\caption{An example of custom font features.}
\label{fig:featurefile}
\hrule
\begin{Verbatim}
\documentclass{article}
\usepackage{fontspec}
\directlua{
    fonts.handlers.otf.addfeature {
        name = "oneb",
        type = "substitution",
        data = {
                ["1"] = "one.ss01",
        }
    }
}
\setmainfont{Vollkorn-Regular.otf}[RawFeature=+oneb]
\begin{document}
1234567890
\end{document}
\end{Verbatim}
\hrule
\end{figure}

\end{document}

% /©
% ------------------------------------------------
% The FONTSPEC package  <wspr.io/fontspec>
% ------------------------------------------------
% Copyright  2004-2019  Will Robertson, LPPL "maintainer"
% Copyright  2009-2015  Khaled Hosny
% Copyright  2013       Philipp Gesang
% Copyright  2013-2016  Joseph Wright
% ------------------------------------------------
% This package is free software and may be redistributed and/or modified under
% the conditions of the LaTeX Project Public License, version 1.3c or higher
% (your choice): <http://www.latex-project.org/lppl/>.
% ------------------------------------------------
% ©/

